\documentclass[10pt]{article}

%\usepackage{fancyhdr}
 
%\pagestyle{headings}
%\markright{John Smith}

\date{}

\usepackage{amsmath}    % need for subequations
\usepackage{graphicx}   % need for figures
\usepackage{verbatim}   % useful for program listings
\usepackage{color}      % use if color is used in text
\usepackage{subfigure}  % use for side-by-side figures
\usepackage[colorlinks=true,citecolor=blue]{hyperref}   % use for hypertext links
\usepackage{lipsum}
\usepackage{url}

\usepackage[margin=1in]{geometry}
\usepackage{lastpage}
\usepackage{graphicx}
\usepackage{balance}
\usepackage{comment}
\usepackage{amssymb,amsmath}
\usepackage{caption}
\DeclareCaptionType{copyrightbox}
\usepackage{subfigure}
\usepackage{enumerate}
\usepackage{color}
\usepackage{titling}
%\usepackage{subcaption}
\newcommand{\figref}[1]{Figure~\ref{fig:#1}}
\newcommand{\tableref}[1]{Table~\ref{tab:#1}}

\newcommand{\compactimg}{\vspace{-12pt}}

\clubpenalty=10000 
\widowpenalty=10000
%\setlength{\parindent}{0cm}



\begin{document}
%\cfoot{\thepage\ of \pageref{LastPage} }
%\rfoot{NR }

%\pagenumbering{gobble}

\begin{table}
\color{blue}
%\color{Emerald}
\begin{tabular*}{\textwidth}{l r}
\large\textbf{DIVERSITY STATEMENT} & 
\hfill \ \ \ \ \ \ \ \ \ \ \ \ \ \ \ \ \ \ \ \
\ \ \ \ \ \ \ \ \ \ \ \ \ 
\large\textbf{NIRANJINI RAJAGOPAL}\\
\hline
\end{tabular*}

\end{table}

%\title{NR}



I have been fortunate to have strong role models growing up, mentors who have supported my personal and professional growth, and opportunities to get me to where I am today. At the same time, through my own experiences and interactions, I have come to recognize barriers and challenges that exist and, and recognize that we have to take a conscious effort to advance equity and inclusion. \\% beyond the classroom, to first provide opportunity %, in order for diversity to flourish. 

%Over the years, I have come to recognize the value of diversity from my personal experience

%As a woman in ECE, 
% During my time at CMU, I have taken several steps to increase participation of women in STEM. % is introducing middle and high school girls to STEM, in a manner where they feel confident. 
% Along with another PhD student, I raised funds, and piloted an outreach program to introduce high school girls to ECE labs in a hands-on manner. I co-led this program for two years. One of my main goals was for students to successfully complete each lab, and develop the confidence to take up further challenges in the field. For this, we designed the labs to be hard, but also provided enough scaffolding. At the end of the labs, the students had to reflect in writing, on what they had learnt. I  volunteered with CMU's Society of Women Engineers middle school days where we introduced ECE labs to middle school students. In addition to generating an interest, my goal through these programs is for the students to develop the confidence that they have the skills to be successful in the field.\\

%Over the years

At Carnegie Mellon, I have been active in outreach and volunteering to increase participation of \textbf{women in STEM}. %to introduce ECE to high school women. 
As an example, along with another PhD student, I started a new program called Mobile Labs to \textbf{introduce ECE to high school girls}. We wrote a grant to \textbf{secure funding} for the program and \textbf{co-led it for two years}. We conducted hands-on labs ranging from programming, microcontrollers, and energy harvesting, to audio processing in a girls' high school in Pittsburgh. I felt it was important that students develop confidence with their first ECE experience. For this, we made sure the labs were designed such that students completed concrete tasks. We also added bonus assignments to challenge the students who wanted to explore further. We actively sought out female undergraduate students to lead the labs. At least \textbf{one of their students joined Carnegie Mellon ECE's summer program} the next year. I have volunteered for two years with the \textbf{Society of Women Engineer's middle school day}, where we introduce middle school girls to various engineering labs. I have formally and informally mentored several female students at CMU, about planning coursework and finding a research group.\\ %One student who joined our group and I mentored for over a year co-authored two research papers. %As a faculty, I would continue to be committed to mentoring increasing 

I found participating in events such as Networking Networking Women (N2women) events at conferences, Graduate Women Gatherings at CMU, and the MIT EECS Rising Stars workshop, to be helpful as they gave me opportunities to interact with female faculty and listen to their perspectives. I organized a N2women event at a conference and created a forum for students, junior and senior faculty to have discussions. As a future faculty, I will take efforts to make myself visible and reach out to female students in K-12 as well as the college level. \\

I have worked with several \textbf{educational outreach programs for K-12 in both India and US} starting from my undergraduate years and involved myself in various activities, ranging from raising funds, teaching, assisting in labs, leading programs, creating instructor manuals, and volunteering. Most of these are avenues where we introduce a new domain to students in a fun, hands-on manner. Through these experiences, I have learned to adapt my approach to the culture, resources available, and the background of the students. \\

%\paragraph{Inclusive Learning Environments.}

%I make a concerted effort to learn about students' backgrounds and interests. In large classes, I conduct ungraded quizzes on the first day of class to assess students prior knowledge, and address any gaps through extra classes and office hours. In smaller classes and while mentoring students for research, I meet students one-on-one early on and understand their background, any specific goals, and their interests, and we discuss various options for moving ahead. 
%One of the challenges I make a concerted effort to \textbf{learn about students' background knowledge} through ungraded quizzes on the first day of class, and through one-on-one meetings. I address any gaps through extra classes and office hours. 

%In smaller classes and while mentoring students for research, I meet students one-on-one early on and understand their background, any specific goals, and their interests. Getting to know students personally,  %For instance, d
%During my recitations in the signals class, where I was explaining a concept through an illustration, one student told me that they were finding it difficult to understand through illustrations and preferred a mathematical approach. Subsequently, I used both methods while explaining concepts and preparing recitation notes that were available to students. 
%In order to
%to not implicitly marginalize 
%support a student with learning disability, I often stayed back for \textbf{extended office hours} and continued to work with them.

Through observing classes in various domains and engaging in pedagogical discussions and readings, I became aware of the effect of classroom climate on student learning. To understand this in more depth, with my peers, we read research papers to understand how factors ranging from the tone in syllabus to the design of peer-based activities impact the classroom climate. 
To make all students feel comfortable in sharing their ideas and asking questions, one strategy I would try in future classes is to first \textbf{break them into smaller groups} to discuss with their group-mates, and then later ask them to share with the class. This would reduce the barrier some students face to speaking up. %I get to know students one-on-one % and I have personally found this effective in seminars and workshops. %One strategy for students to feel free in sharing their ideas is to first break them into smaller groups to discuss with their group-mates, and then later ask them to share with the class. When I have implemented this, it has reduced the barrier some students face to speaking up.\\


I grew up in India, and moved to the US for my graduate school. I have first hand experienced the challenges of transitioning geographically and I am mindful of such transitions while interacting with new students. %continue to mentor new students who move to the US and have discussions with them about navigating the cultural changes. 
I have traveled to over a dozen countries, and through my interactions, I have appreciated differences that comes with culture and geography. and I have also come to recognize that these differences manifest themselves in how students respond in a classroom, in social interactions, and in their comfort in written and oral communication and in their openness.\textcolor{red}{add someting}\\ %In one of my classes, a non-native English speaker I have trained myself to  \\

Diversity for the sake of diversity results in exclusion rather than inclusion. I have experienced this when I am asked to be part of a group just to increase the gender parity. To truly support diversity, I will strive to create inclusive classrooms and research environments, where students are respected for their contributions. \textcolor{red}{add someting}\\
%After mentoring students, I have  one-on-one chat with them and ask them if there is anything I could have done differently to support them better. I have over time, 
%In my role as a Graduate Teaching Fellow at the Eberly Center at CMU, I work with a cohort of peers from several diverse departments at CMU. I experienced the value of the different perspectives we all bring to the discussions, due to our varied disciplinary background. 
%I experienced the value the rich discussions we have had. %when 
%We brainstormed teaching and learning challenges. I also realized how different the classroom interactions are in various deaprtments. 
%We are moving towards interdisciplnary times, where students from various departments are taking courses in computing and engineering. As a faculty, I will make myself aware 
I have made myself aware of implicit biases and take effort to be aware of my biases as well as watch out for biases in social interactions. \textcolor{red}{add someting}%consciously as one step, I made sure to use rubrics as a systematic way to not allow any bias to come in way of grading.  
%As an faculty, I would take efforts to decouple academic credibility from biases by creating \\


Looking ahead, I will continue to use pedagogical strategies to advance equity and inclusion in the classroom. I would create a culture in my research group that is inclusive, open in communication and supports individual students values and needs. I am looking forward to contributing to UCLA's already rich K-12 outreach initiatives, including the Building Engineers and Mentors program at UCLA.   


% At CMU, I have been active in outreach and volunteering to increase participation of women in STEM. %to introduce ECE to high school women. 
% As an example, along with another PhD student, I started a new program called Mobile Labs to introduce ECE to high school girls. We wrote a grant to secure funding for the program and co-led it for two years. We conducted hands-on labs ranging from programming, microcontrollers, and energy harvesting, to audio processing in a girls' high school in Pittsburgh. I felt it was important that students develop confidence with their first ECE experience. For this, we made sure the labs were designed such that students completed concrete tasks, and reflected in writing on what new skills they learned. We also added bonus assignments to challenge the students who wanted to explore further. We actively sought out female undergraduate students to lead the labs. At least one of their students joined CMU ECE's summer program the next year. I have volunteered for two years with the Society of Women Engineer's Middle School day, where we introduce middle school girls to various engineering labs.\\

% I have formally and informally mentored several female students at CMU, about planning coursework, finding a research group, and connected them to research groups that they joined. One students who joined our group and I mentored for over a year co-authored two research papers. I found that participating in seminar events such as Networking for Women in Networking (N2women), Graduate Womens Gathering at CMU, and the MIT EECS Rising Stars workshop, to be helpful as they gave me opportunities to interact with female faculty and listen to their perspectives. I organized a N2women event at a conference and created a forum for students, junior and senior faculty to have discussions. As a future faculty, I will take efforts to make myself visible and reach out and mentor female students in K-12 as well as the college level. \\

%I have been in
%In my experience, economic factors play a major role in contributing to unequal representation of groups in education, due to lack of access to opportunities. 
%I have volunteered with outreach programs in India from my undergraduate years, in raising funds for school supplies, teaching classes for students who had discontinued their schooling temporarily, conducing science events at public schools, and helping create instructor manuals for teaching science through experiments in public schools. Through these experiences have made me aware of the 

% I have volunteered with several outreach programs for K-12 in India from my undergraduate years, in raising funds for school supplies, teaching classes for students who had discontinued their schooling temporarily, conducing science events at public schools, and helping create instructor manuals for teaching science through experiments in public schools. At CMU, I have also volunteered with K-12 for introducing STEM field to elementary m middle school and high school students through the C-MITES program and the ECE Outreach Spark Saturdays program on CMU campus. Through these experiences, I have learned to first understand the culture in the institution, the resources available at hand, understand the background of students and accordingly adapt my approach and teaching style. 
% and  methodology to the culture, resources available, and the background knowledge of students.

%about both the impact of slight interventions, but also made me realize that it is challenging to create sustainable impact and sustainable access to opportunities. In my future role as a faculty, I would like to build sustainable outreach programs over long duration, by working closely with schools and non-profit organizations. \\% to spark curiosity in the students.

%I have volunteered with several educational outreach programs for K-12 in both India and US and involved myself in various activities, ranging from raising funds, teaching, leading programs, creating instructor manuals, and volunteering. Most of these are avenues where we introduce a new domain to students in a fun, hands-on manner. Through these experiences, I have learned to adapt the teaching style to the culture, resources available, and the background knowledge of students.

% Through my role as a graduate teaching fellow at the Eberly Center, both my classroom observation in various domains as well as discussions and readings, I have become aware of the effect of classroom climates on student learning. For instance, To understand this in more depth, along with two of my GTF peers, we did a project and created a short workshop on classroom climate and researched various factors ranging from how the tone in syllabus to the the design of group activities impacts the classroom climate. One indication of an inclusive climate in my experience is one where students feel free to share their ideas and ask questions. One strategy I have found effective for students to feel free in sharing their ideas is to first break them into smaller groups, discuss with their groupmates and then share with the class. When I have implemented this, it has reduced the barrier some students face while speaking up.\\

% While a TA for an undergraduate course, I made myself aware of students with learning disabilities. In order to not implicitly marginalize them but at the same time, in order to to offer them additional support, I often stayed back for extended office hours and continued to work with them.  

%without explicitly asking them if they required help. 
%As a TA I mentored an underrepresented minority, who I continued to mentor informally, and encouraged to enroll in research, they are now joining  and often stayed back after


% I take the time to meet students one-on-one and have on several occasions offered flexible office hours when students are not able to make it to office hours due to personal or medical reasons. While being a TA for the courses, I made notes for all my recitations and made them available to students who were not able to make it in-person for the recitations and who wanted to go over the content at their own pace. 
% Within the  classroom, I adapt my teaching methods to the differences in learning styles of students. %For instance, d
% During my recitations in the signals class, where I was explaining a concept through an illustration, one student told me that they were finding it understand through illustrations and preferred a mathematic approach. Subsequently, I used both methods while explaining concepts and preparing recitation notes that were available to students. 

%I made myself aware of the students in the class with learning disabilities and was flexible in extending my office hours wh

% both , I foudn that some students I was explaining most of the content through diagrams, until a student told me that he prefers if I write out the equation and 

%Looking ahead, I will make sure to have a pedagogical-aw

%The eberly 

%While this is a complex problem, one of the steps I have taken is in working with outreach programs that aim to given students exposure to opportunities that they may not already have. 
%In my experience, one of the factors that contributes to unequal representation of groups in education more broadly is that due to various factors they may not have access to the resources in the first place, or they may not be aware of possible opportunities. 
%Over the years, I have been involved in several educational outreach programs for various causes. 

%Over the years, I have come to recognize the factors that come in way of unequal representation of groups in STEM fields, and have take steps towards towards it through my outreach. I have taken steps towards 

%Over the years, I have come to recognize the value of diversity first-hand, and have come to recognize factors contributing to unequal representation of groups in educational institutions, and have addressed this, primarily through outreach. \\

%I am committed to creating an inclusive culture and contributing towards creating equity. 

%Over the years, I have become aware of individual differences between students, recognize barriers faced that come in way to equality, and 

%Unequal representation of groups consistently in a particular domain or setting occurs due to several factors including social, economic, personal and cultural factors. 

% On one hand, there are social factors such as bias in selection process, and the opportunities offered. On the other hand, there are factors where 
%I am committed to incre
%Through my personal experiences of growing up in India, moving to the US, being a woman in STEM, my interactions at CMU, my involvement in several outreach programs, and working at the Eberly Center at CMU where I became aware from a pedagogy-point of view, the implications of 
%I am committed to taking efforts to understand and address factors and barriers leading to 
% I have come to understand the barriers faced
% Sytematic obstacles:
% Visual, mathematical.\\
% Special needs, made more time for them.\\
% Mentoring 
% %Through my own personal experiences, many conversations, and readings, I have come to recognize 
% "opportunity gap"


%I am committed to addressing this is several ways - first by creating more opportunities to access, second by creating inclusive environments, and by understanding from a research-point-of-view what the factors are.

%\paragraph{Creating Opportunities. }
%\textit{"Students' current level of development interacts with the social, emotional, and intellectual climate of the course to impact learning"} - How Learning Works.\\

%Diversity and equity are complex, and inter-wined 
%Diversity

%opportunities, or theeconomic and financial factors. 
%While this is a complex problem, one of the steps I have taken is in working with outreach programs that aim to given students exposure to opportunities that they may not already have. %Students get interested in a domain based on their exposure to the field. One concrete step towards this is to create programs to introduce stu- dents to new domains outside through outreach. 
%I volunteered for a short time with Agastya foundation, in editing instructor manuals for teaching Physics through experiments to middle school students in rural India. The goal of the program was to inculcate curiosity in students. I subsequently visited the foundation and , that their  interested in science. I volunteered with Youth for Seva back in India to raise funds for basic stationary for public schools, and organized events where my colleagues from work visited the schools and interacted with the students . %with my colleagues. 
%I volunteered at an Oncology hospital teaching classes and organizing activites for kids who had discontinued their schooling. The goal of this intervension was to keep students engaged and interested in different ways. After being involved in several outreach positions, I ahve realized that the real cahllene The challenge I have faced is in sustaining these programs. How can we continue to sustain interest?
%In my experience, economic factors play a major role in contributing to lack of access to opportunities. While this is a complex problem, one of the steps I have taken is in working with outreach programs that aim to given students exposure to opportunities that they may not already have. 
%One such example was that I volunteered with ECE outreach Spark Saturday program, that was aimed towards high school students in Pittsburgh who did not typically have access to computing and engineering facilities. 
 %Through these outreach programs, I developed an. \\


%In order to truly support diversity, we have to create inclusive cultures.
% \paragraph{Creating Inclusive Classrooms. } 
% Students in a class have diverse backgrounds. I have taken efforts to understand students background. Taken extra classes to prepare them better. In smaller classes I have met students one-on-one and create time to meet them. \\
% One strategy I have found effective for students to feel free in sharing their ideas is to first break them into smaller groups, discuss with their groupmates and then share with the class. This reduces the barrier some students face while speaking up. I also made myself aware of implicit biases by readings, workshops at CMU and taking implicit bias tests to become aware  and use gender-neural language. I  make it a point to create rubrics for grading, to ensure fairnes. 

%%Looking ahead, I am committed to contributing to 

%Another challenge is that students assessment of their competence is inter-twined with cultural, social, personal factors. For instance, this especially manifests itself in graduate school and research where there is no clear way to assess progress. 
%Several students, especially from minority groups do not feel confid

%On the research side, I have been interested in 
%On more than one occasion, I have taken the initiative to have challenging conversations with my peers when an inadvertent comment   %, and have learnt about several differences across cultures and geographies. 
%This wide exposure has informed my understanding of cultural differences, and I recognize that geographic relocation can 
%how they manifest in an educational setup. It has also made me aware of mis communications that can occur due to norms being acceptable in one culture being offensive in another. I have on more than one occasttion sat dwn and had discussiona with my peers when I felt that it was important for someone to understand . I w %For instance, cultural differences 

%Finally, I believe role models are important. 
%While being a TA, I have taken 

%For instance, I learnt that attempts to be inclusive towards minorities can make them feel alienated. I have first hand experienced being invited to groups just to increase the gender diversity. In order to support diversity truly, we have to create inclusive cultures. 
%I would strive to create a classroom climate where students feel comfortable in sharing their ideas and having discussions. Often, students are One approach I would try
%One approach that I have taken is creating one-on-one time to meet students and get to know them. 
%\\ I make sure to  

%I have come to recognize that cultural differences impact learning and social interactions. 
%I grew up in India, and moved to the US for my graduate school.  I have traveled to over a dozen countries, and appreciate the differences in cultures that I observed. I have also experienced first-hand how differences in culture and geography manifest themselves in the classroom and social settings in our institutions. %On more than one

%interactions with experienced first-hand how differences in culture manifest themselves in . This wide exposure has informed my understanding of cultural differences, and how they manifest in an educational setup. For instance, cultural differences. Students communicate in several ways including 

 %In order to support diversity truly, we have to create inclusive cultures. I have first hand experienced being invited to groups just to increase the gender diversity. These measures are alienating. At the same time, 

%In my expe
%I have found myself in settings where I feel comfortable in speaking up and sharing my ideas, and also settings where I feel reluctant or anxious. 
%I have found a few approaches to be effective
%In my experience, groups that are underrepresented in a setting are less likely to speak up and share ideas. This hesitation in itself is a barrier. I have found two approaches effective for addressing this in the classroom. First is to have smaller group discussions and then discuss with the larger class. The second is creating a rapport within the classroom and getting to know the students.  As another step in future, I would be mindful of how I phrase my questions. Open-ended questions get more participation.\\
 %lowers that barrier and makes students (and you) feel comfortable speaking up regarding ideas 


%I have also had several positive experiences with groups making a deliberate effort to truly be inclusive.  
%However, the challenge I have faced is in sustaining the 
% One program was Youth for Seva, where we wanted to sustain students interest in education, while they had discontinued their schooling temporarily due to health issues. I volunteered for a short time in creating instructor manuals for middle school Physics including helping create instructor manuals 

%At CMU, I also volunteered with C-MITES, a program to introduce elementary and middle school students to math and science labs in a hands-on fun manner. My goal here was to create an early interest in the field.\\






%Through my own experience and through mentoring female students, I have come to recognize some of the barriers 
%In my experience, another factor that is helpful is role models, and having interactions with other w
%One of the factors I have seen that contributes to unequal representation of women in STEM is the lack of self-efficacy, or the belief that they can success in that domain.  In my experience, this is due to stereotypes in the society that they  

%In my experience, one of the reasons for female students to be less likely to reach out to opportunities in STEM is the ste
%One of the steps I have taken is 

% Through my research I want to create 

% \paragraph{Creating inclusive atmosphere in classroom. }
% In my expe
% I have found myself in settings where I feel comfortable in speaking up and sharing my ideas, and also settings where I feel reluctant or anxious. 
% %I have found a few approaches to be effective
% In my experience, groups that are underrepresented in a setting are less likely to speak up and share ideas. This hesitation in itself is a barrier. I have found two approaches effective for addressing this in the classroom. First is to have smaller group discussions and then discuss with the larger class. The second is creating a rapport within the classroom and getting to know the students.  As another step in future, I would be mindful of how I phrase my questions. Open-ended questions get more participation.\\
%  lowers that barrier and makes students (and you) feel comfortable speaking up regarding ideas 

% \paragraph{Recognizing cultural differences. }
% Common in classrooms.\\
% Having conversations 

% \paragraph{Educating myself on factors contributing to inequity. }
% In addition to these initiatives, I have taken the effort to understand the underlying factors through a combination of my experiences, conversations with others and through reading of research. \\
% My role as a graduate teaching fellow at the Eberly Center has made a tremendous impact in my understanding of several issues through discussions of topics and research. For instance, I learnt about the stereotype threat, and realized that often intentions by instructors to be helpful, by giving extra opportunities, come across negatively by triggering the stereotype threat. The question here was, what can I actually do as an instructor? I came across literature about taking efforts for creating explicitly inclusive atmosphere, right from the design of syllabus. Subsequently in the course I TA-ed, I took the effort to meet students one-one, not only to understand their technical background but to understand their interests but to let them know that I can create time to meet them. Creating time to meet students outside of what is required is a step I have taken. In future teaching, I would like to introduce group activities. Here, I understand the challenge is in creating an overall inclusive atmosphere including peer-peer interactions. I researched this area and found that as an instructor you can still take steps. I will continue to take an approach of thoughtfully designing my courses and research groups, being mindful of these subtle phenomena what we might ignore. \\   

% \paragraph{Creating equity by providing opportunities. }
% One of the factors that contributes to unequal representation of groups in educational in
% In my experience, one of the factors that contributes to unequal representation of groups in education is economic and financial factors. Students get interested in a domain based on their exposure to the field. One concrete step towards this is to create programs to introduce students to new domains outside through outreach. I volunteered for a short time with Agastya foundation, in editing instructor manuals for teaching middle school students Physics through experiments. The goal of the program was to inculcate curiosity and encourage students to continue their education by being interested. I volunteered with Youth for Seva back in India to raise funds for basic stationary for public schools, and organized events to visit the schools with my colleagues. I volunteered at an Oncology hospital teaching classes for kids who had discontinued their schooling. The goal of these intervensions was to keep students engaged and interested in different ways. The challenge I have faced is in sustaining these programs. How can we continue to sustain interest? \\

% In the US, I volunteered with ECE outreach, where we reached out to schools in Pittsburgh, where students were unlikely to have access to engineering fields through their regular schools. Students came to CMU campus and we conducted hands-on ECE labs. This program was successful. However, it was not scalable as students would have to come to campus. As a step towards this, we started a mobile labs program where our goal was to create a truly mobile lab that could be replicated. \\

% \paragraph{Women in STEM}
% We piloted this program in a girls high school. Ran it for two years. \\
% As further steps towards increasing participation of women in STEM, SWE middle school girls. Female students leading the labs, as role-models.\\

% Mentoring female students, brought in female students into the group
% \paragraph{My plans. }
% Inclusive atmosphere in research group - Have open conversations.\\
% I have a strong interest in 

% Access to technology specifically is an economic barrier. Sometimes there aren't enough opportunities for students to be exposed 

% Growing up in India, while I had the fortune of an elite private-school education, it was far from the opportunities available for a typical student. While we cannot %While this is a complex probl
% it was far from what the I saw a wide disparity in 
% technology-oriented fields. The first is economic factors. Though school 



% I have first hand experienced the uphill battle of being a woman in STEM. 
% Personally, I have struggled with 

% Graduate school can get rough. It is unrealistic and impractical to expect alland I am aware of how it can a


% I have taken several steps. I have brought in feamles students into my research group and have mentored them. 

% What do I believe?
% We should respect each person's individual values.
% Each person should have equal opportunity.
% We should understand the underlying factors that are unconscious.
% Let each person be themselves.
% Diverse groups are productive.
% We have to be 

% As woman in STEP, I have faced several situations, where I felt I couldn't be myself,

% where I felt I . I realized that not able to be oneself .

% While I was working in India, there was a physically disabled employee in the manufacturing division who had a . Though 
% I requested the management to fund. I found stores that custom made . I realised that 

% My role as a graduate teaching fellow at the Eberly Center has made a tremendous impact in my understanding of several issues through discussions of topics and research. For isntnace, 
% I am very well aware of the stereotype threat, and have experienced it first hand as a woman in STEM. I did not realize this, unless we started reading papers about stereotype threat and along with another GTF we designed a short seminar on classroom climate, and

% I have been through tough times during my graduate school and sought out help from CMU's counceling and physhiological services. Through many conversations with graduate students, I have 

% How do my experiences make me diverse?
% Culturally, growing up in India, diversity inpersonal values

% Over the yers, I have come to recognize 

% Over the years, I have taken efforts to increase participation of women in STEM.

% I feel economic one of the factors that plays a role in 

% Mentored female students. One of my students, 

% Each student in unique. Different background\\

% Culturally, modesty 

% In my role as a Graduate Teaching Fellow at the Eberly Center at CMU, I have had several opportunities to discuss 
%  As a graduate teaching fellow at the eberly center, I designed a 30 mins seminar with two other graduate teaching fellows on "classroom climate". I researched stereotrype and steoretype threat, syllabus design and how the syllabus can accoutn for diversity, \\
%  Marginalizing centralizing, implicit and explicit.
%  (1) stereotype threat (2) syllabus (3) student-student interaction.\\

% Underrepresented minority, spent time one-one, continued to mentor, research projects, applying for PhD.\\

% Back in India, economic background. Given talks, and met students in public schools.\\

% I make myself visible to female undergraduate students.\\

% Economic status - visit schools 

% I educated myself 
% I have taken online implicit bias tests to become aware of my biases.\\
%  Taken efforts such as anonymysing student names.\\
%  Creating rubrics for grading that are clear.\\

%  Mentored female students.\\





\end{document}

