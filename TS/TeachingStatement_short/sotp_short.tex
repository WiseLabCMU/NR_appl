\documentclass[10pt]{article}

%\usepackage{fancyhdr}
%  \usepackage{fancyheadings}
% \pagestyle{fancy}
% \usepackage[english]{babel}
% \usepackage[utf8]{inputenc}
% \usepackage{fancyhdr}
%\pagestyle{headings}
%\markright{John Smith}

\date{}

\usepackage{amsmath}    % need for subequations
\usepackage{graphicx}   % need for figures
\usepackage{verbatim}   % useful for program listings
\usepackage{color}      % use if color is used in text
\usepackage{subfigure}  % use for side-by-side figures
\usepackage[colorlinks=true,citecolor=blue]{hyperref}   % use for hypertext links
\usepackage{lipsum}
\usepackage{url}
%\usepackage{fancyheadings}
\usepackage[margin=1in]{geometry}
\usepackage{lastpage}
\usepackage{graphicx}
\usepackage{balance}
\usepackage{comment}
\usepackage{amssymb,amsmath}
\usepackage{caption}
\DeclareCaptionType{copyrightbox}
\usepackage{subfigure}
\usepackage{enumerate}
\usepackage{color}
\usepackage{titling}
\usepackage{scrpage2}
%\usepackage{subcaption}
\newcommand{\figref}[1]{Figure~\ref{fig:#1}}
\newcommand{\tableref}[1]{Table~\ref{tab:#1}}

\newcommand{\compactimg}{\vspace{-12pt}}


\ifoot[Teaching Statement: Niranjini Rajagopal]{}
\cfoot[]{}
\ofoot[\pagemark]{\pagemark}

\pagestyle{scrplain}

%\rhead{Your Name Here\\6.xxx\\ \today}
% \rhead{}
% \lhead{}
% \lfoot{Teaching Statement - Niranjini Rajagopal}
% \rfoot{\thepage}
% \renewcommand{\headrulewidth}{0pt}
% \renewcommand{\footrulewidth}{0pt}

\clubpenalty=20000 
\widowpenalty=20000
\setlength{\parindent}{0cm}


\begin{document}
%\pagenumbering{gobble}
%\thispagestyle{empty}
\begin{table}
\color{blue}
%\color{Emerald}
\begin{tabular*}{\textwidth}{l r}
\large\textbf{TEACHING STATEMENT} & 
\hfill \ \ \ \ \ \ \ \ \ \ \ \ \ \ \ \ \ \ \ \
\ \ \ \ \ \ \ \ \ \ \ \ \ \ \ 
\large\textbf{NIRANJINI RAJAGOPAL}\\
\hline
\end{tabular*}

\end{table}


%I enjoy teaching and working with students. I am passionate about education and am looking forward to playing an active role in the personal and professional growth of students. 
%am looking forward to taking on the roles of a teacher and mentor as a faculty . %I enjoy teaching and working with students.    %I am looking forward to taking on the roles of being a teacher and mentor as a faculty. 
%I believe teachers and mentors play an important role in the personal and professional growth of students. 

%as a faculty.

My teaching philosophy is informed by my work at the Eberly Center for Teaching Excellence and Educational Innovation at Carnegie Mellon, by teaching in my discipline, and by a variety of roles I have held in outreach programs. As a \textbf{Graduate Teaching Fellow} for four years at the Eberly Center, I have deepened my understanding of pedagogy through reading books and research papers and participating in discussions on pedagogy every two weeks, and %, and attending over a dozen workshops focused on learning and teaching. 
by consulting for graduate students 
%, I have had the opportunity to see diverse teaching styles 
across various departments in Carnegie Mellon. 
%\textit{"Learning results from what the student does and thinks and only from what the student does and thinks. The teacher can advance learning only by influencing what the student does to learn" - Herbert A. Simon. }
%Learning is a complex process, influenced by several factors including students prior knowledge, 
%I take an evidence-based to understanding student learning and  the factors that influence student learning and methods for effective teaching. I use this understanding to adapt my methods around discipline-specific goals for students, and to create inclusive learning environments. 
My approach is to be aware of the underlying factors that influence student learning, and take an evidence-based approach to teaching. I adapt my methods around discipline-specific goals for students, and strive to create inclusive learning environments. 
\\% and to be cognizant of 



%\paragraph{Student Learning Goals. }
As a teacher in the area of Electrical and Computer Engineering, my broad goals for students are the following. %depending on the level of the course and students, I balance the emphasis across these learning goals.
First, students should develop a \textbf{strong foundational understanding}. Second, they should develop a framework to \textbf{connect various concepts to each other}. This includes connecting concepts across courses, connecting theory and practice, and connecting concepts to other domains. Third, they should develop the ability to \textbf{apply their skills to new problems}, and be able to communicate their approach. \\%Depending on the level of the course and students, I adapt the emphasis across these learning goals.\\


As a teaching assistant (TA) for \textbf{Signals and Systems}, my goal was for students to develop strong fundamentals and an intuition of the mathematical concepts. 
I often designed homework problems by breaking them down into smaller, more manageable sub-problems. %Students said this was helpful since they were able to more easily see connections between the concepts required to solve the entire problem. 
This helped students more easily see connections between the concepts required to solve the entire problem.
I conducted quizzes, and revised my recitations specifically based on their level of understanding. 
%For a particular concept that students struggle with every year, I put in extra effort and attempted different techniques. %I held a special class where we dived deep and spent time in  
When students wrote equations incorrectly, rather than merely correcting them, I spent time with them one-on-one in my office hours and helped them step through the process of interpreting what their equations meant intuitively and in the real-world. 
One assessment of these methods was that the \textbf{students performance} in the exams on a difficult concept for which I had tried several teaching approaches, was \textbf{significantly better than any of the past years}. \\

To help students make connections between theory and practice,  I \textbf{introduced applied problems} in my recitations and in the homework assignments. We \textbf{invited graduate students} from diverse areas to give talks about their research and how it connects to the course. These talks have \textbf{carried forward since then}, and I continue to give a short talk every year.
As a TA for \textbf{Wireless Networks and Applications}, I \textbf{designed new labs} for students to experiment with the physical layer channel characteristics using various wireless platforms. 
For one of the labs, I asked students to design their own experiment, make a hypothesis about what they expect to see, and then explain any differences between their hypothesis and the experiments. %In another lab, students experimented with channel state information. 
Through these labs, students learned to connect the concepts in the class to the real-world. \\% and demonstrated their understanding of how the unpredictable nature of the physical layer requires sophisticated design from the upper layers.\\ 

A gratifying moment for me was when I got an email from a faculty (after being a two-time TA for the signals course)
%then I was a TA for this course in 2012 and 2015. In 2017 I got an email from the faculty 
that there was a \textbf{change in trend within the department} with more undergraduate students taking upper-level signals courses, due to their experience in the signals and systems course. Four of us who have been TAs for this course over the years were acknowledged to have contributed to the change in trend.\\
%\vspace{-12pt}
%\paragraph{Inclusive Learning Environments.}

I make a concerted effort to learn about students' background knowledge through ungraded quizzes on the first day of class, and through one-on-one meetings. I address any gaps through extra classes and office hours. 
Within the  classroom, I \textbf{adapt my teaching methods} to the differences in learning styles of my students. %For instance, d
During my recitations in the signals class, where I was explaining a concept through an illustration, one student told me that they were finding it difficult to understand through illustrations and preferred a mathematical approach. Subsequently, I used both methods while explaining concepts and preparing recitation notes that were available to students. %In order to
%to not implicitly marginalize 
%support a student with learning disability, I often stayed back for extended office hours and continued to work with them.
%Through observing classes in various domains and engaging in pedagogical discussions and readings, I became aware of the effect of classroom climate on student learning. %To understand this in more depth, with my peers, we read research papers to understand how factors ranging from the tone in syllabus to the design of peer-based activities impact the classroom climate. 
To make all students feel comfortable in sharing their ideas and asking questions, one strategy I would try in future classes is to first break them into smaller groups to discuss with their group-mates, and then later ask them to share with the class. This would reduce the barrier some students face to speaking up. % and I have personally found this effective in seminars and workshops. %One strategy for students to feel free in sharing their ideas is to first break them into smaller groups to discuss with their group-mates, and then later ask them to share with the class. When I have implemented this, it has reduced the barrier some students face to speaking up.\\

%While a TA for an undergraduate course, I made myself aware of students with learning disabilities. 


%I frequently check in with research students I mentor on whe
 %While mentoring students for research, I check-in with them regularly about
%When I was a TA for the wireless networking class, I met each student individually during the first week of class and discussed their background courses and their goals for this course. 
%This understanding guided how I worked with them during the course, in office hours, providing feedback on as. 
\paragraph{Research Mentoring. }
Mentoring students for research has been a gratifying experience for me. 
%When I mentor students for research, I start off by defining concrete problems where they can clearly measure their progress and gain confidence to try out harder problems. Over time, we together define research problems based on their interests. %, and at some point they transition to working independently and teaching me. 
%A masters student I mentored contributed to localization research, became interested in \textbf{robotics}, and subsequently we defined an independent project in simultaneous localization and mapping, and eventually joined a robotics startup. \textbf{She co-authored two research papers} published in IPIN 2016 and IPSN 2018 conferences. % and eventually joined a robotics startup. 
 A masters student I mentored on \textbf{localization research} became interested in robotics, worked on localization and mapping algorithms,  and eventually joined a robotics startup. She \textbf{co-authored two research papers} published at the IPIN 2016 and IPSN 2018 conferences.
Another masters student I mentored on \textbf{mobile sensing research} became interested in augmented realty, worked with me in that area, and subsequently joined Apple. He \textbf{co-authored a paper}, which is under review, and was part of our team that won the \textbf{best demo award} at the IPSN 2018 conference.\\ %Occasionally, I also give them feedback on their communication through writing, presentation or explaining concepts on a whiteboard.

I \textbf{led a student seminar} in CyLab at Carnegie Mellon for two years to create a space where graduate students could give practice talks and receive feedback. In this role, I started panel-based seminars to discuss topics such as fellowships, internships, and job search for \textbf{graduate students' overall professional growth}. 



\paragraph{Courses I Can Teach. }
I can teach courses in the general area of signals, communication, and estimation. For instance: 
Signals and Systems; Digital Signal Processing; Statistical Signal Processing; Estimation and Detection; Digital Communication; Wireless Networks and Applications; and Linear Systems. 
%I can also teach embedded systems and control systems.
Seminar courses I can develop and teach are: Cyber-Physical Systems; Mobile Sensing Applications; Localization, Tracking and Mapping; Internet-of-Things. 
These courses would have a mix of theory and implementation. I plan to have a project component and possibly labs in all of my classes. I would introduce discussions on research papers for my graduate classes. 
%UNIV_XYZ
%I would be glad to align my courses with the growing requirements of the new Computer Engineering major offered at UCLA. 
%UNIV_XYZ
%Created seminars on topics I wanted to understand better: peer feedback, and classroom climate\\


%  \paragraph{Outreach. } 
%  At Carnegie Mellon, I have been active in outreach and volunteering to increase participation of \textbf{women in STEM}. %to introduce ECE to high school women. 
% As an example, along with another PhD student, I started a new program called Mobile Labs to \textbf{introduce ECE to high school girls}. We wrote a grant to \textbf{secure funding} for the program and \textbf{co-led it for two years}. We conducted hands-on labs ranging from programming, microcontrollers, and energy harvesting, to audio processing in a girls' high school in Pittsburgh. I felt it was important that students develop confidence with their first ECE experience. For this, we made sure the labs were designed such that students completed concrete tasks. % and \textbf{reflected in writing on their }. 
% We also added bonus assignments to challenge the students who wanted to explore further. We actively sought out female undergraduate students to lead the labs. At least \textbf{one of their students joined Carnegie Mellon ECE's summer program} the next year. I have volunteered for two years with the \textbf{Society of Women Engineer's middle school day}, where we introduce middle school girls to various engineering labs.\\

% I have worked with several \textbf{educational outreach programs for K-12 in both India and US} starting from my undergraduate years and involved myself in various activities, ranging from raising funds, teaching, assisting in labs, leading programs, creating instructor manuals, and volunteering. Most of these are avenues where we introduce a new domain to students in a fun, hands-on manner. Through these experiences, I have learned to \textbf{adapt the teaching style to the culture, resources available, and the background} knowledge of students. \\



\end{document}

