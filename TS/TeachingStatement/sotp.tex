\documentclass[10pt]{article}

%\usepackage{fancyhdr}
 
%\pagestyle{headings}
%\markright{John Smith}

\date{}

\usepackage{amsmath}    % need for subequations
\usepackage{graphicx}   % need for figures
\usepackage{verbatim}   % useful for program listings
\usepackage{color}      % use if color is used in text
\usepackage{subfigure}  % use for side-by-side figures
\usepackage[colorlinks=true,citecolor=blue]{hyperref}   % use for hypertext links
\usepackage{lipsum}
\usepackage{url}

\usepackage[margin=1in]{geometry}

\usepackage{graphicx}
\usepackage{balance}
\usepackage{comment}
\usepackage{amssymb,amsmath}
\usepackage{caption}
\DeclareCaptionType{copyrightbox}
\usepackage{subfigure}
\usepackage{enumerate}
\usepackage{color}
\usepackage{titling}
%\usepackage{subcaption}
\newcommand{\figref}[1]{Figure~\ref{fig:#1}}
\newcommand{\tableref}[1]{Table~\ref{tab:#1}}

\newcommand{\compactimg}{\vspace{-12pt}}

\clubpenalty=10000 
\widowpenalty=10000
\setlength{\parindent}{0cm}



\begin{document}
\pagenumbering{gobble}

\begin{table}
\color{blue}
%\color{Emerald}
\begin{tabular*}{\textwidth}{l r}
\large\textbf{TEACHING STATEMENT} & 
\hfill \ \ \ \ \ \ \ \ \ \ \ \ \ \ \ \ \ \ \ \
\ \ \ \ \ \ \ \ \ \ \ \ \ \ \ 
\large\textbf{NIRANJINI RAJAGOPAL}\\
\hline
\end{tabular*}

\end{table}


I enjoy teaching and working with students.  %I am looking forward to taking on the roles of being a teacher and mentor as a faculty. 
I believe teachers and mentors play an important role in the personal and professional growth of students. I am looking forward to taking on these roles as a faculty.

\paragraph{Teaching Experience:}
At Carnegie Mellon University (CMU), I was a teaching assistant for signals and systems (undergraduate course) for two semesters, wireless networks and applications (undergraduate and graduate course), and a guest lecturer for embedded systems (undergraduate course). I have assisted with ECE labs for high school students, and with science, maths and robotics classes for K-12 students through outreach initiatives. I have mentored masters students on research projects.  

\paragraph{Teaching Plan:}
The courses I can teach are signals and systems;  digital signal processing; statistical signal processing; inference, estimation and detection; probability and random processes; digital communication; wireless networks and applications; linear systems. I can also teach embedded systems and control systems.
Some seminar courses I can teach are - Mobile sensing applications (will include augmented reality); localization and mapping (theory and applications); internet-of-things; cyber-physical-systems; 
I plan to have a project component  and possibly labs in all my classes. I would introduce discussions on research papers for my graduate classes.

\paragraph{Teaching Philosophy: }
I want students to develop the skills to think critically about solving real-world engineering problems. 
Towards this, first, they should develop a strong foundation from first principles. Second, they should develop a framework to connect various concepts to each other. This includes connecting theory to practice, connecting concepts across courses, and connecting concepts to other domains. Third, they should develop the ability to apply their skills to new problems, and be able to communicate their approach. Depending on the level of the course and students, I adapt the emphasis across these learning goals.\\


While teaching signals and systems, my goal was for students to develop a strong foundation and an intuition of the mathematical concepts.
I often designed homework problems by breaking them down into smaller problems. Students said this was helpful since they were able to know how to connect the concepts to solve the problem. 
I conducted frequent quizzes and gave them feedback, and revised my recitations based on the students understanding. 
%To help students develop an intuition of the concepts. I tried different approaches. 
%I found that some while some students understood the concepts easier through illustrations, others found it easier with a purely mathematical approach. I included both approaches in my recitations and office hours. 
When students found equations confusing and wrote them incorrectly, rather than correcting them, I spent time with them one-on-one in my office hours and helped them step-through the process of interpreting what their equations meant intuitively and in the real-world. %This approach helped them understand the concepts better. 
%I also tried an activity in the classroom where students first solved a convolution problem, then worked in pairs and discussed their appraoch then solved another problem individually. We saw an improvement in the class performance in the second problem. 
One assessment of these methods was that the students performance in the midterm exam on a difficult concept for which I had tried several teaching approaches, was significantly better than any of the past years. \\



To help students make connections between theory and practice,  I introduced applied problems in my recitations and the homework assignments. We invited faculty from different areas to talk about their research. These lectures have continued since then, and every year I give a short lecture in the class about how the course content applies to my research.  I was a TA for this course in 2012 and 2015. In 2017 I got an email from the faculty that over the past few years there was a department-wide change in trend with more students taking upper-level signals courses, due to enjoying the signals and systems course. Four of us who have been a TA for this course over the years were acknowledged to have contributed to the change in trend. \\

While being a TA for the wireless networks class, I designed new labs for students to experiment with the physical layer channel characteristics using various wireless platforms. 
For one of the labs, I asked students to design their own experiments, make a hypothesis about what they expect to see, and then perform the experiment and explain any differences between their hypothesis and the real-world experiments. In another lab, students experimented with channel state information. Through these labs, students learnt to connect the concepts in the class to the real-world applications and demonstrated their understanding of how the unpredictable nature of the wireless channel requires sophisticated design from the upper layers of the stack. Two groups of students subsequently defined their course project based on their interest in these initial labs. \\ 

%Student motivation drives what they do. 
When I have flexibility in working with students individually (for small classes and while mentoring on research projects), I first ask them what their goals are and discuss various options for moving ahead. 
%When I was a TA for the wireless networking class, I met each student individually during the first week of class and discussed their background courses and their goals for this course. 
%This understanding guided how I worked with them during the course, in office hours, providing feedback on as. 
When I mentor students for research, I start off by defining concrete problems where they can clearly measure their progress, and gain confidence to try out harder problems. Over time we together define research problems based on their interest, and at some point they transition to working independently and start teaching me. One student who got interested in robotics, subsequently worked on an independent project in simultaneous localization and mapping and eventually joined a robotics startup. She co-authored two research papers. Another student I mentored got interested in augmented reality and took courses in vision and graphics, and subsequently joined Apple. He is a co-author on a paper and was part of our team that won best demo. I also give students feedback on how they communicate in writing, presentation or explaining concepts on a whiteboard.


 \paragraph{Outreach: } 
 At CMU, I have been active in outreach and volunteering to increase participation of women in STEM. %to introduce ECE to high school women. 
As an example, along with another PhD student, I started a new program called mobile labs to introduce ECE to high school girls. I wrote a grant to secure funding for the program and co-led it for two years. We conducted hands-on labs ranging from programming, microcontrollers, energy harvesting, to audio processing in a girls high school in Pittsburgh. I felt it was important that students develop confidence with their first ECE experience. For this, we made sure the labs were designed such that students completed concrete tasks, and reflected on what new skills they learnt. We also added bonus assignments to challenge the students who wanted to explore further. We actively sought out female undergraduate students to lead the labs. This program increased interest in engineering in the school and after the first semester of this program, one of their students joined CMU ECE's summer program. \\

I have worked with several educational outreach programs for K-12 in both India and US and involved myself in various activities, ranging from raising funds, teaching, leading programs, creating instructor manuals and volunteering. Most of these are avenues where we introduce a new domain to students in a fun hands-on manner. Through these experiences, I have learnt to adapt the teaching style to the culture, resources available, and the background knowledge of students. \\

I led a student seminar at CMU for two years to create a space where students could give practice talks and receive feedback. In this role, I started panel-based seminars to discuss topics such as teaching, fellowships, internships and job search for students overall professional growth. 

\paragraph{Graduate Teaching Fellow: }To further my growth as a teacher, I actively seek avenues to improve my own understanding of effective teaching practices from pedagogical experts at CMU. I am among a select group (around 8 students across CMU) of graduate teaching fellows (GTF) at the Eberly Center for Teaching Excellence and Educational Innovation. We have regular meetings where we discuss research in pedagogy and I have attended a dozen seminars in pedagogy as well as created two short seminars on peer feedback and classroom climate, as part of GTF projects. In this role, I consult for graduate student instructors by observing their classes, and providing objective constructive feedback. Interacting with students across various departments in CMU has exposed me to diverse teaching styles and methods. In order to learn about pedagogical research from a computer science and design perspective, I took a course offered by CMU's human computer interaction department on designing large scale peer-learning systems. %. Tattended a course in CMU on the design of large scale peer-learning systems to learn about ways to engage students through peer-based learning in large classes.  
I enjoy taking a research-based approach to teaching and am looking forward to trying out different methods for teaching and engaging students. % working with students and trying out . %trying out new methods and improving them over time based on what is effective for students. %trying out different methods to engage students in the classroom,and facilitate their growth. %Students growth is my priority. % and I will always strive to  . % 

%Created seminars on topics I wanted to understand better: peer feedback, and classroom climate\\
%Combine my interest in research and teaching - by researching in teaching methods, and in teaching students to be researchers. \\
%Continue to mentor undergraduate, and other students from India?


%\small
%\bibliographystyle{alpha}

%\bibliography{sigproc}  % sigproc.bib is the name of the Bibliography in this case

\end{document}
