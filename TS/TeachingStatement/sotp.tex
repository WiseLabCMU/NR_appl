\documentclass[10pt]{article}

%\usepackage{fancyhdr}
 
%\pagestyle{headings}
%\markright{John Smith}

\date{}

\usepackage{amsmath}    % need for subequations
\usepackage{graphicx}   % need for figures
\usepackage{verbatim}   % useful for program listings
\usepackage{color}      % use if color is used in text
\usepackage{subfigure}  % use for side-by-side figures
\usepackage{hyperref}   % use for hypertext links, including those to external documents and URLs
\usepackage{lipsum}
\usepackage{url}

\usepackage[margin=1in]{geometry}

\usepackage{graphicx}
\usepackage{balance}
\usepackage{comment}
\usepackage{amssymb,amsmath}
\usepackage{caption}
\DeclareCaptionType{copyrightbox}
\usepackage{subfigure}
\usepackage{enumerate}
\usepackage{color}
\usepackage{titling}
%\usepackage{subcaption}
\newcommand{\figref}[1]{Figure~\ref{fig:#1}}
\newcommand{\tableref}[1]{Table~\ref{tab:#1}}

\newcommand{\compactimg}{\vspace{-12pt}}

\clubpenalty=10000 
\widowpenalty=10000
\setlength{\parindent}{0cm}



\begin{document}
\pagenumbering{gobble}

%\setlength{\droptitle}{-5em}

\title{{\Large Teaching Statement}}
\vspace{-1em}
%\author{\textit{Niranjini Rajagopal}}
\maketitle

\vspace{-6em}


%I enjoy teaching and mentoring students. 
I am passionate about teaching and education. 



I believe 

 teaching
mentoring students. I This is one of my reasons in addition to research that draws me to an academic path. 

My teaching experience at CMU is from being a teaching assistant twice for signals and systems, an undergraduate level course; the teaching assistant for wireless networks and applications, a mix of undergraduate and graduate level course; guest lecturer in embedded systems course; and through my involvement in various K-12 outreach initiatives.

A common theme in my teaching 


Good technology is built on strong foundations, requires an understanding of the connections between theory and practice, and a broader understanding of how to approach a new problem.  Through my interaction with students, I strive to facilitate this learning process.
 % understanding of the connections between the various components. 
 % in students.



Students learning should be built on strong foundations. % across the courses that I teach. 
When I was teaching the signals and systems course, the biggest challenge was to address the disparity in the mathematical background and learning styles of students. When I taught this class for the second time, I conducted a quiz in the first class to assess the mathematical preparedness of the students. Subsequently, I covered the necessary background material through working with students one-on-one in office hours, recitations and assessing and providing feedback through homeworks. Where possible, I designed the problems to be self-correcting. During recitations and office hours, I asked students what they found most confusing, and went ove that content. I broke down problems into smaller%Another approach I tried was providing them the final solution and ask them to arrive at the solution. This helped them develop their technique to solve the problem. 
Consistently, students struggle when the topic of convolution is introduced in the course. While talking to students, they said that they convolution equation is non-intuitive and confusing. 
My goal was for students to understand the equation and the implications, rather than memorize it. 
One approach I tried was that I held a special recitation just for this topic and 
%I tried a few different techniques to tackle this. I introduced a think-pair share activity, where the students first solved a convolution problem, and then discussed their solutions with their neighbors and then solved another problem individually. I 
%went over what all the incorrect formulae that we came across in their quizzes and homeworks and 
made them think about and interpret what each of them meant mathematically and practically. In the mid-term exams, only 4 out of the 70 students got the convolution equation wrong, and around 55 students solved the problems without major errors - a vast improvement from previous years. 



Students should be able to connect theory and practice. I introduced applied problems from areas of communications, image and speech processing and sensor signals. We also introduced guest lectures where faculty presented their research in context of this class (which has continued since then), and I gave a lecture as well about how the theory applies in my research, and continue to give it every year in this course. This helped students connect the course to the broader ECE. I was a TA for this course in 2012 and 2015. In 2017 I got an email from the faculty that over the past few years there was a department-wide change in trend with more students taking signals courses such as DSP, due to enjoying the Signals and Systems course.
Four of us who have been a TA for this course over the years were acknowledged to have contributed to the change in trend. While guest lecturing and introducing control systems in the embedded systems course, I first ask students 

Should should be able to learn how to approach a problem. 
Wireless course, I mentored studns on their coarse project. Define their own problem,
In office hours, I interacted one-one with students, and would help them arrive at solutions - 
Recitations - different examples - 

Localization course:
Goals: Underlying physics and limits of what is possible. 
Learn the tools 
Learn how to approach a problem - design space is broad.
Learn what components skills they require to solve a problem. 
For instance, in digital comm course, theoretical limits, signal noise model, to learning that there are many ways to design the system and then understand the design factors that play a role.


Learn the tools, mathematics - learn the contraints


While teaching the wireless course, I designed new labs for students to experiment with the physical layer wireless channels and One of my goal for students was to understand the challenges of wireless channels from first principles. I designed new labs for students to experiment with the physical layer signals. 

I asked students to design their own experiments. Design questions they wanted to answer. Make hypothesis and then exprimentally validate. Able to connect the theoretical to practice - think about how to approach a problem.


%Through my teaching, I strive to achieve three goals for students. The first is for them to develop strong foundations and an understanding from first principles. The second learning goal is for them for understand connectons between various components - including theory and practice, and  the ability to break down a problem into smaller components. A related goal is for them to learn that there can be different ways to solve the same problem.


%My goal for students is 

%I enjoy teaching, and thinking about and iterating on how to better design my teaching.  My goal for students in ECE is for them to develop the ability to think about engineering systems with a foundation based on theory and first principles, and rooted in real-world problems and systems. My goal is for them to be able to break down a problem and to be able to think about various components that are require to solve a problem and how they connect to each other. This also requires them to connect the theory and practice and to connect the content in a course to other courses, and disciplines more broadly. \\

%At CMU, I was a Teaching Assistant for the Signals and Systems course two semesters; I conducted recitations, office hours, designed assignments and exams, and the Wireless Networks and Applications course for a semester; I created new labs, held office hours, designed assignments and exams, guided students on course projects and deep dive survey project.\\

In the Signals and Systems course, my goal was for student to develop the mathematical foundation and framework for analyzing signals and systems and then apply the theory to solve problems in different domains. 
%This is the first mathematically intensive core course that undergradautes in CMU ECE take. %As a result, 
%One of the main challenges is that students have varied mathematical backgrounds prior to this course. 
When I was a TA for this course for the first time, I learnt that the biggest challenge was to address the disparity in the mathematical background and learning styles of students. When I taught this class for the second time, I conducted a quiz in the first class to assess the mathematical preparedness of the students. Subsequently, I covered the necessary background material through office hours, recitations and assessing and providing feedback through homeworks. I broke down problems into smaller parts so students could realize where they get stuck. Several students gave feedback that this approach was helpful. Where possible, I designed the problems to be self-correcting. For example, I would provide them the final solution and ask them to arrive at the solution. This helped them develop their technique to solve the problem. %During the term of the course, another approach I attempted to make the mathematical parts more accessible was adopting visual representations of signals. For instance while introducing the the concept of periodic signals in two-dimensions, students were asked to infer the differences in the frequency representation between images with horizontal and vertical stripes of varying widths. 
Consistently, students struggle when the topic of convolution is introduced in the course. %While talking to students, they said that they convolution formula is non-intuitive and confusing. 
My goal was for students to understand the equation and the implications, rather than memorize it. 
I tried a few different techniques to tackle this. I introduced a think-pair share activity, where the students first solved a convolution problem, and then discussed their solutions with their neighbors and then solved another problem individually. I went over what all the incorrect formulae that we came across in their quizzes and homeworks and made them think about and interpret what each of them meant mathematically and practically. In the mid-term exams, only 4 out of the 70 students got the convolution equation wrong, and around 55 students solved the problems without major errors - a vast improvement from previous years. To enable students to make connections between the theory and practice, I introduced applied problems from areas of communications, image and speech processing and sensor signals. We also introduced guest lectures where faculty presented their research in context of this class (which has continued since then), and I gave a lecture as well about how the theory applies in my research, and continue to give it every year in this course. This helped students connect the course to the broader ECE. I was a TA for this course in 2012 and 2015. In 2017 I got an email from the faculty that over the past few years there was a department-wide change in trend with more students taking signals courses such as DSP, due to enjoying the Signals and Systems course.
Four of us who have been a TA for this course over the years were acknowledged to have contributed to the change in trend. \\ 

In contrast to Signals and Systems, the Wireless Networks and Applications course was practical in nature.  %spanning from design of MAC layer protocols, to network traffic to applications of networks in sensing and localization. 
One of my goal for students was to understand the challenges of wireless channels from first principles. I designed new labs for students to experiment with the physical layer signals. Students first had to design their own experiments. For instance, some students opted to experiment with the change in signal with distance, some wanted to test the change in packet reception with crowds, and some wanted to test for variation in data rate across devices and environments. Then students had to describe their hypothesis on the effect of real-world on the signals and sketch out graphs of what they expected to see from the experiments. They then performed the experiments, and presented their observations as well as had to think about and explain the differences between their hypothesis and the real-world experiments. I designed another lab based on CSI monitoring where students learnt about the short-term variations in the channel. Through these experiments early on in the course, students learnt about the fundamental challenges of wireless channels.

%Curiosity and having fun is the first step for a student to get interested in a new domain. However, when students have pre-conceived notions about their abilities to be effective in a new domain, curiosity alone is not enough. 
Over the years, I have worked with several outreach groups. My goal for students in these avenues here has been to inculcate curiosity and the confidence to venture out into a new domain. At CMU, I co-piloted a Mobile Labs outreach program, where we introduced ECE to high school girls in a school where almost none of the girls went to STEM fields. We introduced students to several different areas of ECE with hands-on labs in energy harvesting, LEDs on micro-controllers, audio processing and programming. Here they approached their learning from an applied perspective, and then began to think about how things worked, and the underlying principles. My goal was to design the sessions such that all students complete the lab and could be challenged with harder problems as well. All students were asked to reflect on what new concepts they learnt. After the first year of the program, one of the students enrolled in CMU ECE's summer school. I also worked on a project to teach physics purely with experiments to middle school students in rural India. Here, my big challenge was to design instructor manuals for instructors who would conduct the labs, %who did not have the skills to design the experiment for making students think critically, and the 
and the students did not have the component skills required for middle school physics. Another learning experience for me for to teach robotics and science to primary school students in a hands-on manner. Through these varied experiences, I learnt that the teaching methodology has to be adapted with the background of teachers and students, and the available resources and objectives that are feasible. \\

To further my growth as a teacher, I actively seek avenues to improve my own understanding of effective teaching practices from pedagogical experts at CMU. I have been with the Eberly Center for Teaching Excellence and Educational Innovation in a role of a Graduate Teaching Fellow. Being a part of a team of graduate teaching fellows and interacting with students across various departments in CMU has exposed me to teaching styles and approaches very different from what I am exposed to in engineering.\\

Good technology is built on strong theoretical foundations, requires application of the theory to practice and an understanding of the connections between the various components. I hope that through my interaction with students, I can facilitate this process in students.

\section{Teaching Plan}
most prepared to teach:

I can teach courses in various areas related to signals, communication and embedded systems. The courses I am most prepared to teach are They include mathematical foundations for elctrical engineering, signals and systems, digital communication, wireless communication, digital signal processing, statistical signal processing, probability and random processes, linear systems, embedded systems, estimation and detection theory, random processes in systems. 

Cour
Lab courses, project courses;
seminars;

% and asked the students to infer the differences in the frequency representation based on visual inspection of the images.
%I realized this the first time I TA-ed this course. When I TA-ed this the second time, 
%I designed a quiz that we gave to the students on the first day of class to assess their mathematical background. After finding the areas where students needed help, I conducted extra classes and designed assessements in the first few weeks that prepared the students for the mathematical rigor of the course. 

%have over the years helped..\\ I created notes for this class in 2012, many of which are in use till date.\\
%Talk about revamping several aspects of the class.\\

 %trend has been changing, 

%My third approach was to include both visual and mathematical approaches for solving these problems, as students learned better with one or the other. 
%Students liked this format and over the semesters, the course has evolved to be more applied, and I continue to go back every year and give a talk about how the theory applies to my research. %more faculty, this helped them understand what upper level courses. Subsequently, for three semesters, I present my research in context of the class. I was able \\

%I like to expriment with and try out different approaches and iterate over time, as well as get student feedback on what is working for them. %I re-designed the recitations for this course 
%Finally, some students learn bettern visually, rather than mathematically. I solved problems both ways, and ask them questions such as duration, amplitude. In the mid-term, only 80 - a vast improvement from previous semesters. 


%I believe that to understand and engineer sensing systems for the real-world, 


%and the skills to build systems in practice, discover the challenges and re-iterate. They should also develop the ability to think about how the theory and practice relate to each other. Another goal for students is to be able to connect the theory and practice and to connect how the content relates to other areas, courses, domains. 

% as they may not take the first step at all.  
%Students also often make early judgements about their abilities to be effective in a new domain based on their first few expereinces in the domain. 

% introducing ECE to high school students, and science and maths to middle and primary school students. 
%  and and primary school instruction, introducing them to physics, robotics, and science. Through all these avenues, students first build something, or exeriment with something new, and then begin to think about the underlying principles and componets. 

%Several K-12 outreach,\\
%Main goal has been to develop curiosity \\
%One methods is to design the exercise in a manner where they complete a task, to break the barrier to entry - or any assumptions that the field is hard\\ Different learning styles, important to design 
%Mobile Labs - introduce them to different aspects of energy - physical based - energy harvesting, LEDs on arduino, sound processing - start with what they are already familiar with. They experiement and try out different aspects. \\
%Outreach with Agastya - goal was to inculcate curiosity, science experiment-based introduction to a new concept. Assignments desgined such as they can complete it\\
%Experiment-based learning - wherever possible - experiements develop cuirosity, make them understand how they relate to real-world and flexibility in learning since they can build on...\\
%C-mites robotics for 3rd graders.\\



%to understand the effect of real-world parameters on the signal and packet reception 
%Mention office hours.\\
%In future, I would like to introduce a project in this course, and frame the theoretical. For instance...
%I am interested 

%In contrast to the SandS course, the Wireless Networks and Applications course was more practical.\\
%I approached this problem by helping students engineer systems 



%My first goal for students is to I strive for students to develop a strong mathematical foundation and the ability to 

%Through my teaching 
%I strive to facilitate the process of developing students 
%I enjoy teaching, the process of thinking through the design of courses and seeing students grow as they learn. 
%My teaching philosophy is to facilitate in students the process of thinking critically about systems in the real-world. These systems can be sensing systems, communication systems right from X to Y. Towards this, my goal for students is to understand the mathematical foundations and be able to think about signals and systems from first principles. My second goal is for students to apply theory to practice by building systems/ working hands-on and be able to reason about the challenges and the different possibilities in solving engineering problem. Depending on the course and students, I take either an approach by getting them interested in the practical problem and application, and then diving deeper into the theory, or an approach where they start with the theory and see how they are applied to various domains.
%My goal for students of lower level classes 

%During my time at CMU, I have twice been a Teaching Assistant (TA) for Signals and Systems, an undergraduate level class, been a TA for Wireless Networks and Applications, a mixed class with udergraduate, Masters and PhD students. In addition, over the years I have been part of several outreach programs teaching primary to high school students.\\ %graduate level 

% At CMU, I 

% have twice been a Teaching Assistant for Signals and Systems, an undergraduate level class. My first goal was for students was to learn the mathematical foundations for analysing signals and systems. and to be able to make connections between the theory and real-world systems. 


% While SS the goal was general theory that was applicable to several domains, the next class I TAed was focused on one domain. 
% In order for students to develop an understading of wireless from first priciples, I introduced and designed labs in the course, early on.\\




% Domain: sensing, signals, systems - understanding real-world systems and building them\\

% What skills do I want students to have?\\
% Goals: For students to think critically, first (1) developing the ability to know what component skills they require (2) understanding fundamentals (3) applying to practice and understanding the challenges 
% \begin{enumerate}
% \item Develop a curiosity to understand how the content they are learning applies to everyday life
% \item Understand from first principles
% \item Apply theory to practice by building systems
% \item Be able to assess engineering trade-offs
% \end{enumerate}

% Either an approach where you start with theory and move towards applied, or start with the application to get them interested and get them thinking about the underlying principles.

% Methods:\\
% Introduce applied problems in assignments, give examples from my research, TA\\
% Convolution - explaining what role it has in my current day research, for instance, duration of system impulse response, amplification, etc.\\
% Designing assignments, problems where the math leads up to the application, for instance, is the system invertible, 
% Think-pair-share - understand what is behind the math\\
% Wireless class - designed new labs - physical layer - they can experiment with and understand the unpredictability of the wireless layer to appreciate the need for designing for reliability at the higher layers\\
% Student projects - \\
% Redesigned recitations - \\
% Expose them to applicatins early on.\\

% Assessments:\\
% Quiz - to check how the class was doing and address common problems early-on\\
% Exams in wireless - applied, weekly homework 

% Big challenge:\\
% Working with students with different prior knowledge\\
% SandS: conducted a quiz at the beginning\\
% Wireless grad and undergrad: exams are different - UG tested lesser on applying.\\

% Several K-12 outreach,\\
% Main goal has been to develop curiosity \\
% One methods is to design the exercise in a manner where they complete a task, to break the barrier to entry - or any assumptions that the field is hard\\ Different learning styles, important to design 
% Mobile Labs - introduce them to different aspects of energy - physical based - energy harvesting, LEDs on arduino, sound processing - start with what they are already familiar with. They experiement and try out different aspects. \\
% Outreach with Agastya - goal was to inculcate curiosity, science experiment-based introduction to a new concept. Assignments desgined such as they can complete it\\
% Experiment-based learning - wherever possible - experiements develop cuirosity, make them understand how they relate to real-world and flexibility in learning since they can build on...\\
% C-mites robotics for 3rd graders.\\

% Courses I can teach at undergraduate level - signals and systems, wireless communication, embedded systems, linear systems, control systems,  DSP\\
% Graduate level - estimation, statistical SP or ADSP, wireless SN, wireless n/w\\
% Special courses - localization, sensing systems (theory + applying + building), preliminary robotics course (state estimation + applied)\\
% Special topics (project based): IoT + AR + robotics + SN\\

% Mentoring students:\\
% Two students for 1.5 years, some for a semester. In mentoring students, I enjoy both, teaching as well as getting them introduced to research. \\Challenge is in balancing giving them independence and guidance. \\
% I believe that each student is unique and it takes one-one mentoring to really help push them and..\\
% Initially, teaching, often this is the first time they are doing a task that \\
% My goal is to push them to do more and eventually they teach me\\

% Through all my teaching I have realized that given the time, good amount of independence for students to be creative, and a good amount of guidance, we can help students..\\





\small
\bibliographystyle{alpha}

\bibliography{sigproc}  % sigproc.bib is the name of the Bibliography in this case

\end{document}
