\documentclass[10pt]{article}

%\usepackage{fancyhdr}
 
%\pagestyle{headings}
%\markright{John Smith}

\date{}

\usepackage{amsmath}    % need for subequations
\usepackage{graphicx}   % need for figures
\usepackage{verbatim}   % useful for program listings
\usepackage{color}      % use if color is used in text
\usepackage{subfigure}  % use for side-by-side figures
\usepackage{hyperref}   % use for hypertext links, including those to external documents and URLs
\usepackage{lipsum}
\usepackage{url}

\usepackage[margin=1in]{geometry}

\usepackage{graphicx}
\usepackage{balance}
\usepackage{comment}
\usepackage{amssymb,amsmath}
\usepackage{caption}
\DeclareCaptionType{copyrightbox}
\usepackage{subfigure}
\usepackage{enumerate}
\usepackage{color}
\usepackage{titling}
%\usepackage{subcaption}
\newcommand{\figref}[1]{Figure~\ref{fig:#1}}
\newcommand{\tableref}[1]{Table~\ref{tab:#1}}

\newcommand{\compactimg}{\vspace{-12pt}}

\clubpenalty=10000 
\widowpenalty=10000
\setlength{\parindent}{0cm}



\begin{document}
\pagenumbering{gobble}

%\setlength{\droptitle}{-5em}

\title{{\Large Teaching Statement}}
\vspace{-1em}
%\author{\textit{Niranjini Rajagopal}}
\maketitle

\vspace{-6em}


I enjoy teaching, and thinking about and iterating on how to design my teaching. 

At CMU, I was a Teaching Assistant for the Signals and Systems course two semesters (conducted recitations, office hours, designed assignments and exams), and the Wireless Networks and Applications course (created new labs, held office hours, designed assignments and exams, guiding students on course projects and deep dive survey project).
%I believe that to understand and engineer sensing systems for the real-world, 
My goal for students is for them to develop both - a foundation based on theory or first principles, and the skills to build systems in practice, discover the challenges and re-iterate. They should also develop the ability to think about how the theory and practice relate to each other. Another goal for students is to be able to connect the theory and practice and to connect how the content relates to other areas, courses, domains. 

In the Signals and Systems course, I approached my teaching by first helping the students develop a mathematical foundation and framework for analyzing signals and systems.
%and then applying the theory to solving problems in different application domains. 
This is the first mathematically intensive core course that undergradautes at CMU take. As a result, one of the main challenges is that students have varied mathematical backgrounds prior to this course. I realized this the first time I TA-ed this course. When I TA-ed this the second time, I designed a quiz that we gave to the students on the first day of class to get an assessment on their mathematical background. After finding the areas where students needed help, I conducted extra classes and designed assessements in the first few weeks that prepared the students for the mathematical rigor of the course. Consistently, over the semesters, the topic that students struggle with the most is convolution. While talking to students, they said that they convolution formula is non-intuitive and confusing and they have to memorize it. My goal was for students to learn for themselves what the equations translate to in practice and be able to write it out based on uderstanding, rather than memorization. To tackle this problem, I tried a few different techniques. First, I introduced a think-pair share. Based on what we observed, I held a class where we went over what all the incorrect formual are, and what they would mean practically. Finally, some students learn bettern visually, rather than mathematically. I solved problems both ways, and ask them questions such as duration, amplitude. In the mid-term, only 80 - a vast improvement from previous semesters. My second goal for students in addition to this, was application. Problems from different domains. Introduced guiest lectures. Students like this format and over the semesters, more faculty, this helped them understand what upper level courses. Subsequently, for three semesters, I present my research in context of the class. I was able \\
I TA-ed this course in 2012 and 2015. In 2018 I got an email from the faculty that over the past few years the trend has been changing, four of us -have over the years helped..\\ I created notes for this class in 2012, many of which are in use till date.\\
Talk about revamping several aspects of the class.\\

Mention office hours.\\
In future, I would like to introduce a project in this course, and frame the theoretical. For instance...
I am interested 

In contrast to the SandS course, the Wireless Networks and Applications course was more practical.\\
I approached this problem by helping students engineer systems 



%My first goal for students is to I strive for students to develop a strong mathematical foundation and the ability to 

Through my teaching 
I strive to facilitate the process of developing students 
%I enjoy teaching, the process of thinking through the design of courses and seeing students grow as they learn. 
My teaching philosophy is to facilitate in students the process of thinking critically about systems in the real-world. These systems can be sensing systems, communication systems right from X to Y. Towards this, my goal for students is to understand the mathematical foundations and be able to think about signals and systems from first principles. My second goal is for students to apply theory to practice by building systems/ working hands-on and be able to reason about the challenges and the different possibilities in solving engineering problem. Depending on the course and students, I take either an approach by getting them interested in the practical problem and application, and then diving deeper into the theory, or an approach where they start with the theory and see how they are applied to various domains.
%My goal for students of lower level classes 

%During my time at CMU, I have twice been a Teaching Assistant (TA) for Signals and Systems, an undergraduate level class, been a TA for Wireless Networks and Applications, a mixed class with udergraduate, Masters and PhD students. In addition, over the years I have been part of several outreach programs teaching primary to high school students.\\ %graduate level 

At CMU, I 

have twice been a Teaching Assistant for Signals and Systems, an undergraduate level class. My first goal was for students was to learn the mathematical foundations for analysing signals and systems. and to be able to make connections between the theory and real-world systems. 


While SS the goal was general theory that was applicable to several domains, the next class I TAed was focused on one domain. 
In order for students to develop an understading of wireless from first priciples, I introduced and designed labs in the course, early on.\\




Domain: sensing, signals, systems - understanding real-world systems and building them\\

What skills do I want students to have?\\
Goals: For students to think critically, first (1) developing the ability to know what component skills they require (2) understanding fundamentals (3) applying to practice and understanding the challenges 
\begin{enumerate}
\item Develop a curiosity to understand how the content they are learning applies to everyday life
\item Understand from first principles
\item Apply theory to practice by building systems
\item Be able to assess engineering trade-offs
\end{enumerate}

Either an approach where you start with theory and move towards applied, or start with the application to get them interested and get them thinking about the underlying principles.

Methods:\\
Introduce applied problems in assignments, give examples from my research, TA\\
Convolution - explaining what role it has in my current day research, for instance, duration of system impulse response, amplification, etc.\\
Designing assignments, problems where the math leads up to the application, for instance, is the system invertible, 
Think-pair-share - understand what is behind the math\\
Wireless class - designed new labs - physical layer - they can experiment with and understand the unpredictability of the wireless layer to appreciate the need for designing for reliability at the higher layers\\
Student projects - \\
Redesigned recitations - \\
Expose them to applicatins early on.\\

Assessments:\\
Quiz - to check how the class was doing and address common problems early-on\\
Exams in wireless - applied, weekly homework 

Big challenge:\\
Working with students with different prior knowledge\\
SandS: conducted a quiz at the beginning\\
Wireless grad and undergrad: exams are different - UG tested lesser on applying.\\

Several K-12 outreach,\\
Main goal has been to develop curiosity \\
One methods is to design the exercise in a manner where they complete a task, to break the barrier to entry - or any assumptions that the field is hard\\ Different learning styles, important to design 
Mobile Labs - introduce them to different aspects of energy - physical based - energy harvesting, LEDs on arduino, sound processing - start with what they are already familiar with. They experiement and try out different aspects. \\
Outreach with Agastya - goal was to inculcate curiosity, science experiment-based introduction to a new concept. Assignments desgined such as they can complete it\\
Experiment-based learning - wherever possible - experiements develop cuirosity, make them understand how they relate to real-world and flexibility in learning since they can build on...\\
C-mites robotics for 3rd graders.\\

Courses I can teach at undergraduate level - signals and systems, wireless communication, embedded systems, linear systems, control systems,  DSP\\
Graduate level - estimation, statistical SP or ADSP, wireless SN, wireless n/w\\
Special courses - localization, sensing systems (theory + applying + building), preliminary robotics course (state estimation + applied)\\
Special topics (project based): IoT + AR + robotics + SN\\

Mentoring students:\\
Two students for 1.5 years, some for a semester. In mentoring students, I enjoy both, teaching as well as getting them introduced to research. \\Challenge is in balancing giving them independence and guidance. \\
I believe that each student is unique and it takes one-one mentoring to really help push them and..\\
Initially, teaching, often this is the first time they are doing a task that \\
My goal is to push them to do more and eventually they teach me\\

Through all my teaching I have realized that given the time, good amount of independence for students to be creative, and a good amount of guidance, we can help students..\\





\small
\bibliographystyle{alpha}

\bibliography{sigproc}  % sigproc.bib is the name of the Bibliography in this case

\end{document}
