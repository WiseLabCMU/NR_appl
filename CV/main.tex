%%%%%%%%%%%%%%%%%%%%%%%%%%%%%%%%%%%%%%%%%
% "ModernCV" CV and Cover Letter
% LaTeX Template
% Version 1.1 (9/12/12)
%
% This template has been downloaded from:
% http://www.LaTeXTemplates.com
%
% Original author:
% Xavier Danaux (xdanaux@gmail.com)
%
% License:
% CC BY-NC-SA 3.0 (http://creativecommons.org/licenses/by-nc-sa/3.0/)
%
% Important note:
% This template requires the moderncv.cls and .sty files to be in the same 
% directory as this .tex file. These files provide the resume style and themes 
% used for structuring the document.
%
%%%%%%%%%%%%%%%%%%%%%%%%%%%%%%%%%%%%%%%%%

%----------------------------------------------------------------------------------------
%	PACKAGES AND OTHER DOCUMENT CONFIGURATIONS
%----------------------------------------------------------------------------------------

\documentclass[11pt,a4paper,sans]{moderncv} % Font sizes: 10, 11, or 12; paper sizes: a4paper, letterpaper, a5paper, legalpaper, executivepaper or landscape; font families: sans or roman

\moderncvstyle{classic} % CV theme - options include: 'casual' (default), 'classic', 'oldstyle' and 'banking'
\moderncvcolor{blue} % CV color - options include: 'blue' (default), 'orange', 'green', 'red', 'purple', 'grey' and 'black'

\usepackage{lipsum} % Used for inserting dummy 'Lorem ipsum' text into the template

\usepackage[scale=0.85]{geometry} % Reduce document margins
%\setlength{\hintscolumnwidth}{3cm} % Uncomment to change the width of the dates column
%\setlength{\makecvtitlenamewidth}{10cm} % For the 'classic' style, uncomment to adjust the width of the space allocated to your name

\usepackage{amsmath}
%----------------------------------------------------------------------------------------
%	NAME AND CONTACT INFORMATION SECTION
%----------------------------------------------------------------------------------------

\firstname{Niranjini} % Your first name
\familyname{Rajagopal} % Your last name

% All information in this block is optional, comment out any lines you don't need
%\title{Curriculum Vitae}
%\address{Collaborative Innovation Center, 2224A}{4720 Forbes Ave, Pittsburgh PA 15217}
\mobile{(412) 7087548}
%\phone{(000) 111 1112}
%\fax{(000) 111 1113}
\email{niranjir@andrew.cmu.edu}
\homepage{www.niranjini.com}{www.niranjini.com}
%\homepage{staff.org.edu/~jsmith}{staff.org.edu/$\sim$jsmith} % The first argument is %the url for the clickable link, the second argument is the url displayed in the %template - this allows special characters to be displayed such as the tilde in this %example
%\extrainfo{additional information}
%\photo[70pt][0.4pt]{picture} % The first bracket is the picture height, the second is %the thickness of the frame around the picture (0pt for no frame)

%\quote{Seeking a Job Opportunity or a PhD Placement by the 10th of September - 23 years old}

%----------------------------------------------------------------------------------------

\begin{document}

\makecvtitle % Print the CV title
%----------------------------------------------------------------------------------------
%	EDUCATION SECTION
%----------------------------------------------------------------------------------------
\vspace{-0.3in}
% \section{Research Interests}
% %Embedded Sensing Systems, Estimation and Statistical Signal Processing, Indoor Localization, Mixed Reality
% Embedded Sensing Systems, Mobile Computing, Estimation and Statistical Signal Processing, Indoor Localization
% %, Mixed Reality

%Sensing Systems, Cyber Physical Systems, Wireless Sensor Networks, Indoor Localization, Augmented Reality
%My research interests are in applying signal and information processing to sensing systems, such as those applicable to sensor networks, mobile devices and cyber-physical systems. My thesis is focused on the application of indoor localization systems. 
\section{Education}

\cventry{Jan 2013 onwards}{Carnegie Mellon University (CMU)}{Ph.D.}{Electrical and Computer Engineering}{}{
Thesis: Robust Low Beacon-Density Range-Based Indoor Localization, Placement and Mapping\\
%Thesis: Range-Based Indoor Localization Systems\\
Advisors: Prof. Anthony Rowe, Prof. Bruno Sinopoli. Committee: Prof. Prabal Dutta, Dr. Brent Ledvina
}
%\cventry{2012--2015}{ENSTA Bretagne College - Telcommunications Major}{}{}{}{French Graduate and Post Graduate Engineering School and Research Institute - GRANDE ECOLE\\
%Graduation September 2015}  % Arguments not required can be left empty

\cventry{2011 -- 2012}{Carnegie Mellon University (CMU)}{M.S.}{Electrical and Computer Engineering}{}{}

\cventry{2004 -- 2008}{National Institute of Technology Tiruchirappalli (NITT), India}{B.Tech.}{Electronics and Communication Engineering}{}{}
% \cventry{2011--2012}{Preparatory Classes}{Lyc\'{e}e Mohammed V}{Casablanca}{}{Intensive undergraduate-level preparation in advanced Mathematics and Science for  competitive entrance examinations to French Graduate Engineering Schools}
% \cventry{2009}{High School Diploma in Science and Mathematics}{Lyc\'{e}e Al Khawarizmy}{Casablanca}{}{With honors}


%----------------------------------------------------------------------------------------
%	WORK EXPERIENCE SECTION
%----------------------------------------------------------------------------------------
\section{Awards}
% resume \section{Selected Awards}
\cvitem{2018}{Among MIT EECS Rising Stars}
\cvitem{2018}{First place in Microsoft Indoor Localization Competition in 3D Infrastructure-based category}
\cvitem{2018}{Best Demo Award, International Conference on Information Processing in Sensor Networks (IPSN)}
%resume 
\cvitem{2017}{Ben Taskar Memorial Best Poster Award, TerraSwarm Annual Meeting}
\cvitem{2016 -- 2017}{Samsung Ph.D. Fellowship (among 5 students in US), for Internet of Things area}
\cvitem{2015 -- 2016}{Carnegie Mellon William S. Dietrich II Presidential Ph.D. Fellowship}
\cvitem{2015}{First place in Microsoft Indoor Localization Competition in 2D Infrastructure-based category}
%resume 
%resume 
\cvitem{2014}{Fourth place in Microsoft Indoor Localization Competition in 2D Infrastructure-based category}
\cvitem{2014 -- 2015}{Travel Grants: NSF for SenSys 2015 and MobiCom 2014; ACM SIGBED for CPS Week 2014}
\cvitem{2013 -- 2014}{Carnegie Institute of Technology Dean's Tuition Fellowship}
\cvitem{2014}{Networking Networking (N2) Women Young Researcher Fellowship award for CPS Week}
%resume 
\cvitem{2011}{Narotam Sekhsaria Foundation Scholarship for Higher Studies, India (among 12 students in India)}
%resume 
\cvitem{2011}{J N Tata Endowment Scholarship for Higher Studies, India}
%resume 
\cvitem{2003 -- 2008}{National Talent Search Examination Scholarship (NTSE), awarded by the National Council of Education, Research and Training (NCERT), India}



% \section{Research at CMU}
% \cventry{2014 onwards}{Indoor localization \textit{(Focus of thesis)}}{}{}{}
% {\begin{itemize}
% %\item Designing, implement algorithms for data fusion of multi-sensor localization technologies 
% \item Designed and implemented localization algorithms by integrating information from range-based beacons, coverage models of beacons, floor plan geometry, inertial sensors, magnetic field, to make the systems robust to non line-of-sight signals, operate with low-density beacons, and for instant location acquisition
% \item Developed tool for automating beacon placement given a floor plan, and designed and implemented algorithm for pedestrian-aided beacon-mapping algorithm, using Simultaneous Localization and Mapping
% \item Designed algorithm and system using beacons and magnetic field for persistent multi-user augmented reality on mobile devices
% \item Evaluated performance of designed systems with real world deployments with both ultrasonic beacons and ultra-wideband beacons. 
% \end{itemize}}

% \cventry{2013 -- 2014}{Visible light communication (VLC)}{}{}{}
% {\begin{itemize}
% \item Designed a communication system between LED lights and rolling-shutter based camera devices.
% \item Extended to a hybrid communication scheme for smartphones and low-power embedded tags.
% %\item Extending VLC applications to Wireless Sensor Networks (low-power wake-up, synchronization)
% \end{itemize}
% }


% \cventry{2012 -- 2013}{Ambient electric and magnetic field sensing systems}{}{}{}
% {\begin{itemize}
% \item Designed a magnetic field embedded sensing system for contactless sensing of electrical appliances.
% \item Designed a hardware-based clock tuning circuit, which synchronizes using external ambient electric field.
% \end{itemize}}
% \cventry{Summer 2012}{Modeling of environmental sensor data}{Summer internship}{Mentor: Prof. Rohit Negi}{}
% {\begin{itemize}
% \item Statistical signal processing-based parametric modeling of environmental sensor data indoors for prediction, and actuation for smart buildings.
% \end{itemize}}
% \cventry{2011 -- 2012}{Three-phase energy meter design and data modeling}{}{}{}
% {\begin{itemize}
% \item Designed embedded software for a real-time wireless three-phase energy meter for a micro-grid in Haiti.
% %\item Modeled interventions in supply and demand to improve operations of the rural micro-grid
% \end{itemize}}

% ---------------------------------------------------------------------- %
% ---------------------------------------------------------------------- %

% \section{Research at CMU}
% \cventry{2014 onwards}{Indoor localization \textit{(Focus of thesis)}}{}{}{}
% {\begin{itemize}
% %\item Designing, implement algorithms for data fusion of multi-sensor localization technologies 
% \item Designed and implemented localization algorithms by integrating information from range-based beacons, coverage models of beacons, floor plan geometry, inertial sensors, magnetic field, to make the systems robust to non line-of-sight signals, operate with low-density beacons, and for instant location acquisition.
% \item Developed tool for automating beacon placement given a floor plan, and designed and implemented algorithm for pedestrian-aided beacon-mapping algorithm, using Simultaneous Localization and Mapping.
% \item Designed algorithm and demonstrated system for persistent multi-user augmented reality on mobile devices using beacons and magnetic field. 
% \item Evaluated performance of designed systems with real world deployments with both ultrasonic beacons and ultra-wideband beacons. 
% \item \textit{Ultrasonic-beacon work is now spun out as a start-up from our group, Yodel Labs}.
% \item Designed system proposal; Demonstrated prototype for firefighter localization with UWB beacons and visual inertial odometry.
% \end{itemize}}

% \cventry{2013 -- 2014}{Visible light communication (VLC)}{}{}{}
% {\begin{itemize}
% \item Designed a communication system between LED lights and rolling-shutter based camera devices (\textit{One of the earliest published works/prototypes in overhead LED lights-unmodified camera communication}).
% \item Extended to a hybrid communication scheme for smartphones and low-power embedded tags.
% %\item Extending VLC applications to Wireless Sensor Networks (low-power wake-up, synchronization)
% \end{itemize}
% }


% \cventry{2012 -- 2013}{Ambient electric and magnetic field sensing systems}{}{}{}
% {\begin{itemize}
% \item Designed a magnetic field embedded sensing system for contactless sensing of electrical appliances.
% %\item Designed a hardware-based clock tuning circuit, which synchronizes using external ambient electric field.
% \end{itemize}}
% \cventry{Summer 2012}{Modeling of environmental sensor data}{Summer internship}{Mentor: Prof. Rohit Negi}{}
% {\begin{itemize}
% \item Statistical signal processing-based parametric modeling of environmental sensor data indoors for prediction, and actuation for smart buildings.
% \end{itemize}}
% \cventry{2011 -- 2012}{Three-phase energy meter design and data modeling}{}{}{}
% {\begin{itemize}
% \item Designed embedded software for a real-time wireless three-phase energy meter for a micro-grid in Haiti 
% %\item Modeled interventions in supply and demand to improve operations of the rural micro-grid
% \end{itemize}}

% -------
\section{Publications}
\cvitem{IPSN '18}{Niranjini Rajagopal, Patrick Lazik, Nuno Pereira, Sindhura Chayapathy, Bruno Sinopoli and Anthony Rowe, \textbf{Enhancing Indoor Smartphone Location Acquisition using Floor Plans}, The 17th International Conference on Information Processing in Sensor Networks, 2018, Porto, Portugal}

\cvitem{RTAS '17}{Adwait Dongare, Patrick Lazik, Niranjini Rajagopal, Anthony Rowe, \textbf{Pulsar: A Wireless Propagation-Aware Clock Synchronization Platform}, 23rd IEEE Real-Time and Embedded Technology and Applications Symposium, Pittsburgh, Pennsylvania, April, 2017}

\cvitem{IPIN '16}{Niranjini Rajagopal, Sindhura Chayapathy, Bruno Sinopoli, Anthony Rowe,  {\textbf{Beacon Placement for Range-Based Indoor Localization}}, The 7th International Conference on Indoor Positioning and Indoor Navigation. October, 2016, Madrid, Spain}

\cvitem{SenSys '15}{Patrick Lazik, Niranjini Rajagopal, Oliver Shih, Bruno Sinopoli, Anthony Rowe, \textbf{ALPS: A Bluetooth and Ultrasound Platform for Mapping and Localization}, The 13th ACM Conference on Embedded Networked Sensing Systems. November, 2015, Seoul, South Korea}

\cvitem{IPSN '15}{Lymberopoulos et al., \textbf{A Realistic Evaluation and Comparison of Indoor Location Technologies: Experiences and Lessons Learned}, ACM/IEEE 14th International Conference on Information Processing in Sensor Networks, Seattle, Washington, April 13th, 2015}

\cvitem{RTAS '15}{Patrick Lazik, Niranjini Rajagopal, Bruno Sinopoli, Anthony Rowe, \textbf{Ultrasonic Time Synchronization and Ranging on Smartphones}, 21st IEEE Real-Time and Embedded Technology and Applications Symposium, Seattle, Washington, April 13th, 2015}

\cvitem{VLCS '14}{Niranjini Rajagopal, Patrick Lazik, Anthony, Rowe, \textbf{Hybrid Visual Light Communication for Cameras and Low-Power Embedded Devices}, 1st ACM Workshop on Visible Light Communication Systems, Sep 7, 2014, Maui, Hawaii}

\cvitem{IPSN '14}{Niranjini Rajagopal, Patrick Lazik, Anthony Rowe, \textbf{Visual Light Landmarks for Mobile Devices}, ACM/IEEE International Conference on Information Processing in Sensor Networks, 2014, Berlin Germany}

\cvitem{RTSS '13}{Maxim Buevich, Niranjini Rajagopal, Anthony Rowe, \textbf{Hardware Assisted Clock Synchronization for Real-Time Sensor Networks}, IEEE Real-Time Systems Symposium, Vancouver, CA 2013}

\cvitem{ICCPS '13}{Niranjini Rajagopal, Suman Giri, Mario Berges, Anthony Rowe, \textbf{A Magnetic Field-based Appliance Metering System}, The 4th ACM/IEEE International Conference on Cyber-Physical Systems, Apr. 8th, Philadelphia, USA 2013}

\cvitem{VLSID '09}{Ramasamy, S., B. Venkataramani, R. Niranjini, and K. Suganya. \textit{100KHz-20MHz Programmable Subthreshold $G_m-C$ Low-Pass Filter in $0.18 \mu m$ CMOS}. In 2009 22nd International Conference on VLSI Design, pp. 105-110. IEEE, 2009}

% ---------------------------------------------------------------------- %
% ---------------------------------------------------------------------- %
\section{Patents}
%\cvitem{2017}{\textbf{Method and Apparatus for Locating a Mobile Device}. Patrick Lazik, Niranjini Rajagopal, Oliver Shih, Anthony Rowe, Bruno Sinopoli - US Patent App. 15/171,958, 2016}
\cvitem{2017}{\textbf{Method and Apparatus for Locating a Mobile Device within an Indoor Environment}. Patrick Lazik, Niranjini Rajagopal, Oliver Shih, Anthony Rowe, Bruno Sinopoli - US Patent 9,766,320, 2017}
% \section{Grant Writing}
% \cvitem{2017}{\textbf{Method and Apparatus for Locating a Mobile Device within an Indoor Environment}. Patrick Lazik, Niranjini Rajagopal, Oliver Shih, Anthony Rowe, Bruno Sinopoli - US Patent 9,766,320, 2017}
% ---------------------------------------------------------------------- %
% \section{Involvement with Grants and Funding Agencies}
% \cventry{2017}{NIST}{An Infrastructure-Free Localization Systems for Firefighters}{}{}
% {\begin{itemize} 
% \item Contributed to ideas and the main technical section of the grant
% \item Attended kickoff meeting and annual review with demo. Contribute to reports
% \end{itemize}}

% \cventry{NSF SBIR}{}{}{}{}{
% \begin{itemize}
% \item Contributed to technical sections of grant
% \end{itemize}}

% \cventry{NSF}{BIC}{}{}{}{
% \begin{itemize}
% \item Contributed to ideas and the main technical section of the grant
% \end{itemize}}

% \cventry{Industry}{Texas Instruments}{}{}{}{
% \begin{itemize}
% \item Contributed to technical sections of automatic infrastructure-mapping proposal
% \item Interact with TI over regular calls
% \end{itemize}}

% \cventry{Industry}{Samsung}{}{}{}{
% \begin{itemize}
% \item Contributed technical content to grant: Sensor Fusion for Indoor Localization 
% \item Interacted with Samsung  on calls
% \end{itemize}}

% \cventry{Fellowship}{Samsung}{}{}{}{
% \begin{itemize}
% \item Received industry fellowship and interated with IoT group at Samsung during fellowship year
% \end{itemize}}

% \cventry{2013-2017}{SRC}{Terraswarm Research Center}{}{}{
% \begin{itemize}
% \item Part of multi-university research center. Attended annual review, had posters and demos at the review. Interacted with other universities and representatives from funding agencies during review meetings.
% \end{itemize}}

% \cventry{2018 onwards}{SRC}{CONIX Research Center}{}{}{
% \begin{itemize}
% \item Part of multi-university research center. Attended annual review, gave a talk at the review. Interacted with other universities and representatives from funding agencies during review meeting.
% \end{itemize}}
% ---------------------------------------------------------------------- %



\section{Industry Experience}
\cventry{Summer 2015}{Apple Inc.}{Wireless Location Team}{Mentor: Dr. Brent Ledvina}{}
{\begin{itemize} 
\item %Worked on early experiments, prototype, preliminary modeling of a new wireless ranging technology.
Worked on the earliest experiments, prototype and preliminary modeling of time-of-flight RF ranging technology in the location team, and was selected to present the work and \emph{demonstrate a prototype} to Craig Federighi, SVP of Software Engineering, Apple. Subsequently, this resulted in product impact.
%\item \emph{Selected to present} and demonstrate prototype to Craig Federighi, SVP of Software Engineering, Apple.
\end{itemize}}

\cventry{Summer 2013}{Texas Instruments, Dallas}{Embedded Processing Team}{Mentors: Dr. Srinath Hosur, Dr. Ariton Xhafa}{Hybrid communication over power line and WiFi}{
\begin{itemize}
\item Analyzed the feasibility of integrating wireless and power line communication technologies. 
\item Designed and simulated co-existence of both technologies at the MAC and PHY layer.
\end{itemize}}

\cventry{Aug'09 -- Jul'11}{Signals \& Systems India Pvt. Ltd., Chennai, India}{}{ Role: R\&D Engineer}{Project: Embedded products for power sector, with focus on embedded software}{
\begin{itemize}
\item Designed energy metering products (Reference energy meters, multi-function transducers). Involved in entire product development process from customer specification to field deployment in collaboration with hardware, production and customer support teams. %Worked closely with the hardware and production teams.  
\item Implemented DLMS communication stack for metering applications.
\item Revamped factory calibration processes, reducing the production line time by 60\%.
%\item Mentored new R\&D engineers and trained customer support engineers
% . I worked on early versions (LPC2378, ADE7758-based) of the Reference Energy Meters and the Multi Function Transducers that you can see here. I was part of the product development process from customer specifications to deployment and worked closely with the hardware and production teams. I had a chance to implement the DLMS communication stack for metering applications, which was quite new for the metering products during that time.
\end{itemize}}

\cventry{Jun'08 -- Jul'09}{Analog Devices Inc., Bangalore, India}{}{Role: IC Design Engineer in the SHARC DSP Group}{Project: SHARC 2146x-2148x verification}{
\begin{itemize}
\item Designed and implemented test plan for the Variable Instruction Set Architecture, verified Core and IOP modules at the RTL and Gate Level.
\end{itemize}}



\section{Demonstrations}

%resume 
\cvitem{NIST '18} {Niranjini Rajagopal, John Miller, Anh Luong, Anthony Rowe, \textbf{An Infrastructure-Free Localization System for Firefighters}, Public Safety Broadband Stakeholder Meeting, organized by NIST, San Diego, June, 2018 (not peer-reviewed)}

\cvitem{IPSN '18}{Niranjini Rajagopal, John Miller, Krishna Kumar, Anh Luong, Anthony Rowe, \textbf{Demo Abstract: Welcome to My World: Demystifying Multi-user Augmented Reality with the Cloud}, The 17th International Conference on Information Processing in Sensor Networks, 2018, Porto, Portugal \textcolor{red}{(Best Demo Award)}}


%resume 
\cvitem{IPIN '16}{Niranjini Rajagopal, Sindhura Chayapathy, Bruno Sinopoli, Anthony Rowe, \textbf{A Toolchain for Beacon Placement for Range-Based Indoor Localization}, The 7th International Conference on Indoor Positioning and Indoor Navigation. October, 2016, Madrid, Spain}

\cvitem{SenSys '15}{Patrick Lazik, Niranjini Rajagopal, Oliver Shih, Bruno Sinopoli, Anthony Rowe, \textbf{Demo Abstract: Where Am I And Where Are The Walls?}, The 13th ACM Conference on Embedded Networked Sensor Systems, 2015, Seoul, South Korea}

\cvitem{IPSN '14}{Niranjini Rajagopal, Patrick Lazik, Anthony Rowe, \textbf{Demo Abstract: How Many Lights do You See?}, in Proceedings of the 13th ACM/IEEE International Conference on Information Processing in Sensor Networks, Berlin, Germany,2014}

\cvitem{ICCPS '13}{Niranjini Rajagopal, Suman Giri, Mario Berges, Anthony Rowe, \textbf{Demo Abstract: Magnetic Field-based Appliance Metering System}, The 4th ACM/IEEE International Conference on Cyber-Physical Systems, Apr. 8th, Philadelphia, USA}

% ---------------------------------------------------------------------- %
% ---------------------------------------------------------------------- %



%resume 
\section{Workshops}
%resume 
\cvitem{USC '18}{Mixed Reality Workshop, organized by CONIX Research Center, at Institute for Creative Technologies, University of Southern California, Aug 2018}
%resume 
\cvitem{UCLA '18}{Enhanced Situational Awareness Workshop, organized by CONIX Research Center, at University of California Los Angeles, Aug 2018}
%resume 
\cvitem{VLCS '14}{1st ACM Workshop on Visible Light Communication Systems (VLCS), in conjunction with MobiCom, Sep 7, 2014, Maui, Hawaii. \textbf{(Poster: Is There a Place for VLC in Wireless Sensor Networks?}}
%resume 
\cvitem{UMich '14}{Indoor localization Workshop, organized by the TerraSwarm Research Center, at University of Michigan
%resume 
\textbf{(Poster: Visible Light Landmarks for Phones and Low-Power Sensors)}}
%resume 
\textbf{}\cvitem{CMU '14}{Open Building Automation Systems Workshop, organized by DOE Building Technologies Office Project, at CMU \textbf{(Poster: Solid-State Lighting for Sensor Mapping)}}


% ---------------------------------------------------------------------- %
% ---------------------------------------------------------------------- %

\section{Talks}
% 
\cvitem{CyLab}{\textbf{Augmented Reality meets Internet-of-Things}, CyLab Partners Conference, Pittsburgh, Sep 2018}
\cvitem{CONIX}{\textbf{Mobile Augmented Reality}, CONIX Annual Review, Pittsburgh, Sep 2018}
\cvitem{Magic Leap}{\textbf{Towards Location-Aware Computing}, Magic Leap, Seattle Sep 2018}
\cvitem{Amazon}{\textbf{Towards Location-Aware Computing}, Amazon, Seattle, Sep 2018}
\cvitem{Intel}{\textbf{Towards Location-Aware Computing}, Intel Labs, Santa Clara, Sep 2018}
\cvitem{COMPASS}{\textbf{The Current Status of Research on Mobile Location Aware Technology}, Conference on Mobile Position Awareness Systems and Solutions, San Francisco Exploratorium, Sep 2018}
% 
\cvitem{KTH}{\textbf{Where am I? A Sensor-Fusion Approach to Indoor Localization}, ETH Zurich, June 2018}
% 
\cvitem{ETH}{\textbf{Where am I? A Sensor-Fusion Approach to Indoor Localization}, KTH Royal Institute of Technology Stockholm, June 2018}
% 
\cvitem{IPSN}{\textbf{Enhancing Indoor Smartphone Location Acquisition using Floor Plans}, The 17th International Conference on Information Processing in Sensor Networks, 2018, Porto, Portugal}
% 
\cvitem{IPIN}{\textbf{Beacon Placement for Range-Based Indoor Localization}, The 7th International Conference on Indoor Positioning and Indoor Navigation (IPIN), Oct 2016}
% 
\cvitem{ETH}{\textbf{Automatic Placement and Mapping of Beacon-based Localization Systems}, ETH Zurich, Oct 2016}
% 
\cvitem{CyLab}{\textbf{Grappling with Billions of Devices - A Step Towards Spatially-Aware IoT}, CyLab Partners Conference, CMU, Sept 2016}
% 
\cvitem{Samsung}{\textbf{Sensor Fusion and Automatic Infrastructure Mapping for Indoor Localization Systems}, Samsung, San Jose, March 2016}
% 
\cvitem{MSR}{\textbf{Smartphone-based Indoor Localization}, Microsoft Research, Bangalore, India, May 2015}
\cvitem{IISc }{\textbf{Smartphone-based Indoor Localization}, Robert Bosch Centre for Cyber-Physical Systems, Indian Institute of Science, Bangalore, India, May 2015}
\cvitem{SII}{\textbf{Smart Lighting: Technology and Applications for Building Automation}, Carnegie Mellon Smart Infrastructure Institute (SII), Dec 2014}
\cvitem{VLCS}{\textbf{Hybrid Visible Light Communication for Cameras and Low-Power Embedded Devices}, 1st ACM Workshop on Visible Light Communication Systems Workshop (VLCS), Sep 2014}
\cvitem{IPSN}{\textbf{Visual Light Landmarks for Mobile Devices},
13th ACM/ IEEE International Conference on Information Processing in Sensor Networks (IPSN), Apr 2014}
\cvitem{ICCPS}{\textbf{A Magnetic Field-based Appliance Metering System},
4th ACM/IEEE International Conference on Cyber-Physical Systems (ICCPS), Apr 2013}



% ---------------------------------------------------------------------- %
% ---------------------------------------------------------------------- %


% \section{Courses at CMU}
% \cvitem{}{Digital Signal Processing, Advanced Digital Signal Processing, Linear Systems, Smart Grid and Future Energy Systems, Wireless Communication, Information Theory, Convex Optimization, Networked Cyber Physical Systems, Estimation Detection and Identification, Machine Learning, Entrepreneurship and Technology Innovation Management.}



\section{Teaching}
\cventry{Spring 2017}{Wireless Networks and Applications, Teaching Assistant}{ECE, CMU} {}{}{}
\cventry{Spring 2015}{Signals and Systems, Teaching Assistant}{ECE, CMU} {}{}{}
\cventry{Fall 2012}{Signals and Systems, Teaching Assistant}{ECE, CMU} {}{}{}
%resume 
\cventry{Jan'15 onwards}{Eberly Center for Teaching Excellence and Educational Innovation, Graduate Teaching Fellow}{} {}{}{
%resume 
\begin{itemize}
%resume 
\item Teaching peer consultant, to support the professional development of the CMU graduate student community. % by providing classroom observations, facilitating microteaching workshops, providing feedback on teaching strategies, and facilitating early-course feedback focus groups in classroom. 
%resume 
\item Gain deeper professional development in pedagogical research and learning science through consultation training, pedagogical reading and discussions.
%resume 
\end{itemize}}
\cventry{Jan'15 onwards}{Enrolled in Future Faculty Program, CMU}{} {}{}{
%resume 
Attended seminars in effective course design and pedagogy by the Eberly Center for Teaching Excellence and Educational Innovation, received feedback on teaching in workshop and classroom setting, and applied to teaching philosopy project and syllabus design project.
}
%resume 
\section{Students Mentoring}
%resume 
\cventry{May '17-'18}{Enhancing augmented reality on iOS with sensors}{Krishna Kumar}{MS ECE, CMU}{}{}
%resume 
\cventry{Jan-May '17}{Fusion of IMU with acoustic ranging}{Nikhil Choudhary}{BS ECE, CMU}{}{}
%resume 
\cventry{May '15-'16}{Integration of floor plan for range-based localization}{Sindhura Chayapathy}{MS ECE, CMU}{}{}

% 
%\cventry{2015 onwards}{Research mentor for students}{ECE, CMU} {}{}{
% \cvitem{COMPASS '18}{f}{f}{s}
% \begin{itemize}
% \item MS student Krishna Kumar Reghu Kumar (Sensor Fusion and Augmented Reality Applications on iOS)
% \item MS student Sindhura Chayapathy (Integration of floor plan for range-based localization)
% \item BS student Nikhil Choudhary (Fusion of IMU with acoustic ranging).
% \end{itemize}}


\section{Professional Service and Leadership}
\cvitem{2017}{Student volunteers coordinator, CPS Week 2017, Pittsburgh, PA}
\cvitem{2015}{Student organizer, Workshop on Wearable Systems and Applications, co-located with MobiSys}
\cvitem{2015}{Shadow Technical Program Committee member, IPSN}
\cvitem{2014}{Organizer, N2 (Networking Networking) event for women researchers at CPS Week}
\cvitem{2014 -- 2016}{Organizer, CyLab student seminar series}
%resume 
\cvitem{2014 onwards}{Reviewer - IEEE Wireless Communications Magazine '14, Conference on Decision and Control '14 (External), CoNEXT '15 (External), IEEE ICC Optical Wireless Comm. Workshop '16, IEEE Transactions on Signal Processing '16, IEEE Transactions on Mobile Computing '16 \& '17, Symposium on Wireless Personal Multimedia Communications '17, IEEE Wireless Communications Magazine '17, IEEE Vehicular Technology Conference '18, IEEE Communications Letters '18, IEEE Sensors Journal '18, IEEE Access '18, Transations on Sensor Networks '18}
\cvitem{2012 -- 2013}{Vice President, ECE Masters Students Advisory Council}
  \vspace{-0.1in}
%resume 
\section{K-12 Outreach at CMU}
%resume 
\cvitem{2014 -- 2016}{Co-chair ECE Outreach Mobile Labs, CMU. Expanded ECE Outreach program to high-schools. Initiated pilot program with Oakland Catholic Girls High School, Pittsburgh in Spring 2015}
%resume 
\cvitem{2013 -- 2017}{Teaching Assistant, ECE Outreach Spark Saturdays Program, CMU. Assist in introductory electrical and computer engineering classes for grade 9-12 students}
%resume 
\cvitem{2013 -- 2014}{Volunteer, High School Days, Society of Women Engineers}
%resume 
\cvitem{Aug'12 -- Jul'14}{Volunteer, Carnegie Mellon Institute for Talented Elementary and Secondary Students (C-MITES). Assisted in mathematics, robotics and science hands-on classes for grade 1-5 students}
%\cvitem{May'13 -- Dec'13}{Volunteer, Agastya International Foundation, India. Edited instructor manuals for teaching physics through experiments for grade 6-8 students}
%\cvitem{Jul'08 -- Aug'09}{Volunteer, Youth for Seva Foundation, India. Conducted classes for children in Kidwai Memorial Institute of Oncology.}

%resume 
% 
% \section{Talks}
% % 
% \cvitem{CyLab}{\textbf{Augmented Reality meets Internet-of-Things}, CyLab Partners Conference, Pittsburgh, Sep 2018}
% \cvitem{CONIX}{\textbf{Mobile Augmented Reality}, CONIX Annual Review, Pittsburgh, Sep 2018}
% \cvitem{Magic Leap}{\textbf{Towards Location-Aware Computing}, Magic Leap, Seattle Sep 2018}
% \cvitem{Amazon}{\textbf{Towards Location-Aware Computing}, Amazon, Seattle, Sep 2018}
% \cvitem{Intel}{\textbf{Towards Location-Aware Computing}, Intel Labs, Santa Clara, Sep 2018}
% \cvitem{COMPASS}{\textbf{The Current Status of Research on Mobile Location Aware Technology}, Conference on Mobile Position Awareness Systems and Solutions, San Francisco Exploratorium, Sep 2018}
% % 
% \cvitem{KTH}{\textbf{Where am I? A Sensor-Fusion Approach to Indoor Localization}, ETH Zurich, June 2018}
% % 
% \cvitem{ETH}{\textbf{Where am I? A Sensor-Fusion Approach to Indoor Localization}, KTH Royal Institute of Technology Stockholm, June 2018}
% % 
% \cvitem{IPSN}{\textbf{Enhancing Indoor Smartphone Location Acquisition using Floor Plans}, The 17th International Conference on Information Processing in Sensor Networks, 2018, Porto, Portugal}
% % 
% \cvitem{IPIN}{\textbf{Beacon Placement for Range-Based Indoor Localization}, The 7th International Conference on Indoor Positioning and Indoor Navigation (IPIN), Oct 2016}
% % 
% \cvitem{ETH}{\textbf{Automatic Placement and Mapping of Beacon-based Localization Systems}, ETH Zurich, Oct 2016}
% % 
% \cvitem{CyLab}{\textbf{Grappling with Billions of Devices - A Step Towards Spatially-Aware IoT}, CyLab Partners Conference, CMU, Sept 2016}
% % 
% \cvitem{Samsung}{\textbf{Sensor Fusion and Automatic Infrastructure Mapping for Indoor Localization Systems}, Samsung, San Jose, March 2016}
% % 
% \cvitem{MSR}{\textbf{Smartphone-based Indoor Localization}, Microsoft Research, Bangalore, India, May 2015}
% \cvitem{IISc }{\textbf{Smartphone-based Indoor Localization}, Robert Bosch Centre for Cyber-Physical Systems, Indian Institute of Science, Bangalore, India, May 2015}
% \cvitem{SII}{\textbf{Smart Lighting: Technology and Applications for Building Automation}, Carnegie Mellon Smart Infrastructure Institute (SII), Dec 2014}
% \cvitem{VLCS}{\textbf{Hybrid Visible Light Communication for Cameras and Low-Power Embedded Devices}, 1st ACM Workshop on Visible Light Communication Systems Workshop (VLCS), Sep 2014}
% \cvitem{IPSN}{\textbf{Visual Light Landmarks for Mobile Devices},
% 13th ACM/ IEEE International Conference on Information Processing in Sensor Networks (IPSN), Apr 2014}
% \cvitem{ICCPS}{\textbf{A Magnetic Field-based Appliance Metering System},
% 4th ACM/IEEE International Conference on Cyber-Physical Systems (ICCPS), Apr 2013}
% ---------------------------------------------------------------------- %
% ---------------------------------------------------------------------- %

%\cvitem{2013}{Membership chair, Indian Graduate Student Association}
%\section{Student Leadership Activities}



% \section{Experience}

% \cventry{March--September 2015}{Final Project Internship}{\textsc{CGG (EX CGG VERITAS)} Massy, France}{\small ENSTA Bretagne}{}{Developping a "Control and Command System" demonstrator (Hardware and Software) for Land and marine acquisition surveys with special reference to communication infrastructure in remotes areas and Data Centric Publish Subscribe (DCPS) middleware. 
% \textit{Matlab/C++/Java/MOOS-IvP environments}}

% \cventry{July--September 2014}{Assistant Engineer Internship}{\textsc{Lab-STICC} UMR CNRS 6285}{\small ENSTA Bretagne}{}{Empirical Mode Decomposition and Blind Source Separation Methods for Antijamming. 
% \textit{Matlab/C environment}}

% \cventry{October--June 2014}{Smart Wheelchair Control using EEG}{\small ENSTA Bretagne}{}{}{Developing a Low Cost EOG Signal Interface for a Smart Wheelchair with High Accuracy and Reliability. \textit{Matlab/Python environment}}

% \cventry{July--August 2013}{Industrial Placement}{Hypios SAS}{Paris}{}{Database and Customer Relationship Management.}

% \cventry{2012}{Supervised Personal Work}{}{Lyc\'{e}e Mohammed V}{}{Implementing a Best-Routing ADA Algorithm based on Paris Metro. \textit{ADA environment}}


% %----------------------------------------------------------------------------------------
% %	COMPUTER SKILLS SECTION
% %----------------------------------------------------------------------------------------

% \section{Computer skills}

% \cvitem{Basic}{VHDL, UML}
% \cvitem{Intermediate}{ADS, \LaTeX, OpenOffice, Linux, Microsoft Windows, OpenGL, OpenCV, ADA, HTML}
% \cvitem{Advanced}{\textsc{java}, \textsc{Matlab}, Python, C/C++}


% %----------------------------------------------------------------------------------------
% %	LANGUAGES SECTION
% %----------------------------------------------------------------------------------------

% \section{Languages}
% \begin{small}
% \cvitemwithcomment{Arabic}{Native Speaker}{}
% \cvitemwithcomment{French}{Near Native}{excellent command}
% \cvitemwithcomment{English}{Near Native}{excellent command}
% \cvitemwithcomment{Chinese}{Intermediate}{good working knowledge}
% \cvitemwithcomment{Japanese}{Intermediate}{good working knowledge}
% \end{small}


% %----------------------------------------------------------------------------------------
% %	INTERESTS SECTION
% %----------------------------------------------------------------------------------------
% %\bigskip

% \section{Interests}

% \renewcommand{\listitemsymbol}{-~} % Changes the symbol used for lists

% \cvlistdoubleitem{Trecking}{Robotics (NAO Junior Developer)}
% %\cvlistdoubleitem{Robotics}{}


%----------------------------------------------------------------------------------------
%	COVER LETTER
%----------------------------------------------------------------------------------------

% To remove the cover letter, comment out this entire block

%\clearpage

%\recipient{HR Departmnet}{Corporation\\123 Pleasant Lane\\12345 City, State} % Letter recipient
%\date{\today} % Letter date
%\opening{Dear Sir or Madam,} % Opening greeting
%\closing{Sincerely yours,} % Closing phrase
%\enclosure[Attached]{curriculum vit\ae{}} % List of enclosed documents

%\makelettertitle % Print letter title

%\lipsum[1-3] % Dummy text

%\makeletterclosing % Print letter signature

%----------------------------------------------------------------------------------------

\end{document}