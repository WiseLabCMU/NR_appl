\documentclass[10pt]{article}

%\usepackage{fancyhdr}
 
%\pagestyle{headings}
%\markright{John Smith}

\date{}

\usepackage{amsmath}    % need for subequations
\usepackage{graphicx}   % need for figures
\usepackage{verbatim}   % useful for program listings
\usepackage{color}      % use if color is used in text
\usepackage{subfigure}  % use for side-by-side figures
\usepackage[colorlinks=true,citecolor=blue]{hyperref}   % use for hypertext links
\usepackage{lipsum}
\usepackage{url}

\usepackage[margin=1in]{geometry}
\usepackage{lastpage}
\usepackage{graphicx}
\usepackage{balance}
\usepackage{comment}
\usepackage{amssymb,amsmath}
\usepackage{caption}
\DeclareCaptionType{copyrightbox}
\usepackage{subfigure}
\usepackage{enumerate}
\usepackage{color}
\usepackage{titling}
%\usepackage{subcaption}
\newcommand{\figref}[1]{Figure~\ref{fig:#1}}
\newcommand{\tableref}[1]{Table~\ref{tab:#1}}

\newcommand{\compactimg}{\vspace{-12pt}}

\clubpenalty=10000 
\widowpenalty=10000
%\setlength{\parindent}{0cm}



\begin{document}
%\cfoot{\thepage\ of \pageref{LastPage} }
%\rfoot{NR }

%\pagenumbering{gobble}

\begin{table}
\color{blue}
%\color{Emerald}
\begin{tabular*}{\textwidth}{l r}
\large\textbf{RESEARCH STATEMENT} & 
\hfill \ \ \ \ \ \ \ \ \ \ \ \ \ \ \ \ \ \ \ \
\ \ \ \ \ \ \ \ \ \ \ \ \ 
\large\textbf{NIRANJINI RAJAGOPAL}\\
\hline
\end{tabular*}

\end{table}
 
I am interested in designing  embedded sensing systems and inference algorithms for modern cyber-physical systems (CPS). 
I leverage theoretical frameworks from signal processing and estimation and apply them to the complexities that arise from real-world sensing systems. 
I work close to the physical layer, and I understand the practical constraints, the interaction between signals and systems from first principles, and the theoretical tools relevant to the problem. 
The balance between systems and theory in my work positions me extremely well to tackle the important challenges in designing systems that are practical, cost-effective, efficient, and reliable. My thesis demonstrates this approach applied to several indoor localization applications, ranging from extremely precise mobile phone localization that can be used for navigation and mobile augmented reality, to more extreme systems designed for firefighting scenarios wherein multiple incomplete sources of data need to be opportunistically fused together. 

% \paragraph{Old Intro: }
% My research lies at the intersection between embedded systems, estimation theory, and signal processing applied to Cyber-Physical System (CPS) applications.  From a sensing and estimation perspective, the challenge in CPS is in dealing with heterogeneity in sensing across a large number of devices and the lack of a systematic way to go from sensor data to inference. I tackle this challenge by developing an understanding of the practical constraints, the theoretical tools relevant to the problem and a first principles understanding of the interaction between sensors, signals and systems. I then apply these constraints to the design of embedded sensing systems, system models and inference algorithms. My thesis demonstrates this approach applied to several indoor localization applications ranging from extremely precise mobile phone ranging that can be used for way-finding and mobile augmented reality to more extreme systems designed for firefighting scenarios where multiple incomplete sources of data need to be opportunistically fused together.  

%My research lies at the intersection of theory and systems applied to modern Cyber-Physical System (CPS) design. 
%My research focuses on the design of embedded sensing systems and inference algorithms applied to modern cyber-physical systems (CPS).
%My interest lies in using analytical design approaches based on first principles while still capturing the complexities of system-level implementation.  
%I take an approach that leverages theoretical frameworks from signal processing and estimation and apply them to the complexities that arise from real-world sensing systems. 
%Solving the challenges that lie at the interface between systems and theory is not just critical for designing practical systems, but required if these are to become cost-effective, efficient and reliable. 
%I have a strong interest in a systematic approach to design and I appreciate the complexities of system-level implementation.  
%I work across a breadth of areas ranging from signal processing and estimation to system level design of sensing, communication and computing.
%I build systems that are practical and design information processing algorithms that are applicable to a broad class to systems. %I achieve this by understanding the physical layer 
%I design embedded sensing systems, experiment with them in the real world and design system models and inference algorithms that convert sensor data to meaning.
% 
%I draw from theoretical frameworks in signal processing and estimation and apply them to real-world distributed sensing and communication applications. 
%of not just designing practical systems, but ones that are cost-effective, efficient and reliable. %From a sensing and estimation perspective, the challenge in modern CPS is in dealing with the diversity in sensor types, devices and applications
%My approach is to understand the practical constraints, the theoretical tools relevant to the problem and the interaction between sensors, signals and systems from first principles.
%I then apply these to design novel embedded sensing platforms and information processing algorithms. 
%I jointly design the sensor front-end, system models, the networked embedded system and the information processing algorithms. I believe my strong interest in systematic, rather than ad-hoc, approach to design combined with my appreciation for system level implementation and my breadth of knowledge, ranging from Signal processing to system level design of the sensing, communication and computing platforms, puts me in a perfect place to tackle the important challenges of designing of practical, cost-effective, yet reliable and efficient CPS. 
%I apply these to build systems that are practical and design information processing algorithms that are applicable to a broad class to systems. %I have demonstrated my approach with indoor localization for my dissertation. I have deployed and proven several indoor localization systems in the real-world.
 


%embedded sensing systems, inference algorithms and tools. 


%I then apply these constraints to the design of embedded sensing systems, system models and inference algorithms. 

%My research lies at the intersection between embedded systems, estimation theory, and signal processing applied to Cyber-Physical System (CPS) applications.  
%From a sensing perspective, the challenge in CPS is the heterogeneity in sensors across a large number of devices and the lack of a systematic way to go from sensor data to inference. 
%I tackle this by understanding the practical constraints, the theoretical tools relevant to the problem and the interaction between sensors, signals and systems.
%I then apply these constraints to the design of embedded sensing systems, system models and inference algorithms. My thesis demonstrates this approach applied to several indoor localization applications ranging from extremely precise mobile phone ranging that can be used for way-finding and mobile augmented reality to more extreme systems designed for firefighting scenarios where multiple incomplete sources of data need to be opportunistically fused together.  

%I have demonstrated my approach with indoor localization for my dissertation. I have deployed and proven several indoor localization systems in the real-world. 

\paragraph{Impact. }
My research has resulted in publications at IPSN '14 (cited 130+ times), IPSN '18, SenSys '17, RTAS '15, RTAS '17, RTSS '13, ICCPS '13, IPIN '16 and VLCS '14. My work has been demonstrated live at four conferences, been deployed in more than two dozen environments, received two patents, won the international Microsoft Indoor Localization competition twice, received a best demo award, spawned a startup, and led to funding from NSF, SRC, NIST, and industry. %This work is being applied to indoor navigation, mobile persistent augmented reality, firefighter localization, and asset mapping applications.


\section{Indoor Localization Research}


Several solutions, like GPS, have emerged in the past decade for localizing mobile devices, yet there is still no pervasive and accurate system for indoor environments.  It is unclear if a single technology will emerge that can solve all indoor localization problems, because of the wide variability in application requirements and the constraints imposed by indoor spaces.  Across solutions, there is a trade-off between cost due to additional infrastructure and environment-specific configuration and setup, and performance, in terms of accuracy and time taken for the location to converge. I  take an alternate approach where I derive insights that cut across the sensing, embedded systems, and estimation stack. I jointly explore emerging localization technologies, the location estimation algorithms as well as tackle the challenges of system setup. 
%My work has explored new techologies, location estimation algorithms as well as tackled the challenges of system setup. 
%My approach is to design emerging technologies and the location estimation algorithms and system setup tools to enable them at scale in the real-world. 
%To this end, my work has explored multiple emerging technologies in terms of estimating location as well as the challenges of configuration and setup.


\paragraph{Emerging Technologies. }

I designed one of the earliest visible light communication (VLC) systems to send data from overhead LED lights to phones. The phones localize themselves to be in proximity of the lights detected. The challenge in building this communication channel is that the camera frame rate is much lower than the light's operating frequency. 
To overcome this, my insight was to exploit the low-level rolling-shutter effect of camera sensors to capture frequencies higher than the frame rate, %a time-varying light signal as a spatially varying image 
 and to use the exposure and focus control as filters to improve the signal-to-noise ratio \cite{rajagopal2014visual, rajagopal2014demonstration}. I extended this to design a novel hybrid communication system for camera and photodiodes \cite{rajagopal2014hybrid}. 
%The proposed approaches generalizes to any LED-camera communication system and have been cited 170+ times. Subsequently the field of VLC has grown significantly in both academia and industry.

I contributed to building an ultrasonic time-of-flight (ToF) platform that localizes unmodified mobile devices \cite{rtas-alps-platform, lazik2015alps,lazik2015alpsdemo}.  I have worked with Apple on emerging WiFi ToF, with Texas Instruments on emerging Bluetooth Low Energy 5 ToF, and with commodity ultra-wideband ToF systems. I realized that ToF ranging is promising for localization and is getting integrated into future smart devices.  However, they face challenges in scaling up, with respect to location estimation and system setup. I then tackled these challenges. 


\paragraph{Estimation Algorithms. } 
%I designed sensor fusion methods to acquire location and orientation accurately under practical conditions. 
Location estimation is done in two stages. First, an initial location is acquired, and then the location is updated over time with new measurements. 
%Location estimation has two stages - initialization or acquisition, and location updates when we have a prior. 
Acquiring the initial location with ToF systems is challenging in the real world, since we often have only few line-of-sight (LOS) ranges and several incorrect non-line-of-light (NLOS) ranges. Further, we don't have ways to acquire orientation instantly. 

I designed a novel solver that localizes with just two LOS beacons, rather than three (for 2D localization), and maintains the same performance with high NLOS \cite{rajagopal2018enhancing}.  %This tackles the practical problem at scale of having too few line-of-sight ranges and having incorrect non-line-of-light ranges. 
My main insight was that we can integrate the floor plan while solving and use the absence of measurements from beacons as useful information. 
This significantly reduces the amount of infrastructure and increases robustness of the system. 
%I implemented this in our system that won the Microsoft Indoor Localization Competition in 2015.
This solver implemented on our ultrasonic platform won the Microsoft Indoor Localization Competition in 2015. %A second practical problem is that beacon-based systems cannot estimate device orientation. 

Acquiring orientation on phones is hard since the magnetic field is unreliable indoors. To solve this problem, 
%Applications like navigation and augmented reality require orientation in addition to location. 
I designed a novel approach %using beac. % to acquire orientation. %The challenge is that the magnetic field is unreliable indoors. 
%My insight was that 
where we %we could 
crowd-source a dense magnetic field map with mobile pedestrians by fusing data from beacons, on-board camera and inertial sensors. 
Subsequently, we leverage this map as a reference for instantly acquiring orientation upon startup. %without requiring a user to walk around before they can start using the application. 
Using this concept, we built an end-to-end multi-user persistent augmented reality (AR) system \cite{mobileAR}. This work won best demo award at IPSN 2018 \cite{rajagopal2018welcome}. I extended this work to support continuous location updates. This system won the Microsoft Indoor Localization competition in 2018 with ultra-wideband beacons. 

These approaches are implemented on our ultrasonic localization system that spawned into a startup. A pilot has been deployed for AR-based product finding in a retail store. 

\paragraph{Tools for Scalable Setup of Systems.  }
Beacon setup at scale lacks a systematic method, is laborious and time-consuming, and does not adapt over time. This prevents the scaling of systems in a cost-effective manner.

To bring order to the deployment chaos, I designed systematic beacon placement algorithms \cite{rajagopal2016beacon}. 
My first insight was to use the floor plan geometry in a clever way to reduce the number of beacons compared to conventional placement. 
My second insight was to quantify the quality of a beacon placement by adopting the Cramer-Rao lower bound on the location estimate, which is captured by the geometry of beacons. I built on these concepts and implemented beacon placement algorithms in a toolchain where system installers can specify accuracy and coverage requirements for a floor plan and obtain a placement. 

While working on this, I found that we could apply theory in computation geometry to solve these problems in a rigorous manner.  
I started collaborating with Prof. Jie Gao from Stony Brook University and her student. 
We mathematically formulated the beacon placement problem and proposed algorithms with provable guarantees \cite{beaconplacementtheory}.  

%A complementary problem in beacon setup is to first place them and then infer their locations.  
To solve the complementary problem where we first place beacons and then map their locations, I built on algorithms in robotics and designed a pedestrian-based simultaneous localization and mapping (SLAM) algorithm \cite{mobileAR}. We implemented this in the real world for mapping ultra-wideband beacons and for asset tracking with BLE tags. 

%I believe such tools for setting up systems are necessary to scale them up in a cost-effective manner.

% \paragraph{Ongoing Research in Indoor Localization. }
% %My vision for mobile indoor localization is that the future we will have a variety of technologies that operate together 
% In contrast to mobile indoor localization for augmented reality, another direction I pursued was localization under challenging environments.  
% I led the proposal of a firefighter localization project, which is funded by National Institute of Standards and Technologies (NIST). Our proposed system is a combination of fixed beacons on firetrucks that are automatically mapped using our SLAM algorithm; firefighters with wearable devices that range to each other; and estimation algorithms based on mobile network localization. We are exploring the possibility of exit signs as potential locations for low-power beacons in emergencies. 
% In future, I would integrate a network of drones for increasing resilience. 

% After working with several localization technologies, I was interested in analyzing how the localization stack I have developed would change if we have to re-design these systems to be robust to certain types of physical layer attacks on the ranging signals.
% I have started exploring this problem space with a security research group. % and am analyzing how the localization stack I have developed would change to be robust to certain attack models.\\

% \paragraph{Security. }
% These future smart devices necessarily have to be secure. I am interested in understanding how multiple sensor feeds can make systems more robust to physical layer attacks. The intuition is that different types of sensors have different models and we can check for consistencies among them. %systems are not usable unless secure.\\
% %After working in the area of localization, I started thinking about what are the security implications in localization. 
% As a start in this direction, I have started exploring this problem space with a security research group and am analyzing how the localization stack I have developed would change to be robust to certain attack models. 
%I believe I can draw on my experience in system modeling to re-designing systems for security. 
%Some questions I am pursuing are - how would the beacon deployment in a building change
%to guarantee robustness against attacks? Can we use consistency between various sensors and beacon measurements to detect attacks? 
%how does the device discovery and MAC layer change based on security properties of the physical layer? 

\paragraph{Other Areas in CPS. }
I have also contributed to other areas of CPS:
time-synchronization \cite{buevich2013hardware, dongare2017pulsar} and non-intrusive electrical load
monitoring \cite{rajagopal2013magnetic, rajagopal2013demo}.

\section{Future Work}
%In the near-term, I would apply my indoor localization expertise to new areas. 
In the near-term, I want to apply my expertise in localization to solve problems in other domains (re-write).   
%solve problems more broadly in sensing, communication and designing more secure systems. 
In the long-term, I want to explore new paradigms to go from sensor data to inference in resource-efficient ways, applied to emerging CPS.

\paragraph{Localization for First Responders . }
%My vision for mobile indoor localization is that the future we will have a variety of technologies that operate together 
%In contrast to mobile indoor localization for augmented reality, another direction I pursued was localization under challenging environments.  
%I led the proposal for a firefighter localization research project, which is funded by National Institute of Standards and Technologies (NIST). This problem is challenging since we cannot rely on any existing infrastructure. Our proposed method uses beacons deployed on firetrucks and wearable devices on firefighters, in combination with mobile network localization algorithms. I am exploring several research directions ahead, with challenges in sensing, system design and algorithms, in order for the approach to scale across buildings and be reliable in the worst firefighting scenarios. 
%In addition, a future direction is to integrate a network of drones for increasing resilience. 
I led the proposal for a firefighter localization project, which is funded by National Institute of Standards and Technologies (NIST). My vision is to design a robust system for localizing first responders and building occupants in the worst fires without any relying on existing infrastructure. % by opportunistic use of available sensors. % This problem is challenging since we cannot rely on any existing infrastructure. 
As a first step, our proposed method uses beacons on firetrucks, wearable devices on firefighters, a long-range low-power wireless network, and mobile network localization and mapping algorithms. 
My next steps are to evaluate the feasibility of various technologies under smoke, develop methods to train low-accuracy inertial sensors using vision-based high accuracy inertial sensors, integrate infrared sensors and integrate a network of drones for increasing resilience. %mobile network localization algorithms. As next steps, I %I am exploring several research directions ahead, with challenges in sensing, system design and algorithms, in order for the approach to scale across buildings and be reliable in the worst firefighting scenarios. 
%In addition, a future direction is to integrate a network of drones for increasing resilience. 

%For instance, we could place
%Our proposed system is a combination of fixed beacons on firetrucks and firefighters with wearable devices. The challenge in estimation is that we have insufficient and sporadic sources of information. 
%As an extension to the proposed method, on the system side we are exploring the possibility of placing low-power beacons within exit signs in buildings, and on the algorithms side we are exploring using visual-inertial odometry to train  training modelscommodity inertial sensors using higher-fidelity sensors. 
%Our proposed system is a combination of fixed beacons on firetrucks that are automatically mapped using our SLAM algorithm; firefighters with wearable devices that range to each other; and estimation algorithms based on mobile network localization. We are exploring the possibility of exit signs as potential locations for low-power beacons in emergencies. 

\paragraph{Security and Localization. }
Indoor localization is imminent. However, current systems can easily be attacked and locations can be spoofed. It is critical to make these systems secure at all layers of the stack. Secure location awareness will be key to enable device discovery and device-to-device interaction in the internet-of-things with heterogeneous untrusted devices.
%Security and localization are siloed domains traditionally. However, there is an opportunity in working across the domains to make localization secure, and to leverage localization and sensing for enabling secure interaction among physically co-located devices. 
%To start thinking about localization from a security-point-of-view, 
I have started working on problems in secure localization with a security research group. % 
%n order to  started collaborating As a start, I visited a security research group for a week and started 
%re-designing looking at how the ToF localization stack I have designed would change for these systems to be robust to certain physical layer attacks. 
I am analyzing how to de-design the location estimation algorithms and the beacon placement algorithms for robustness to certain physical layer attacks. Preliminary thoughts are to use multiple sensor sources and check for consistency among them since they have different physical layer properties. More broadly, I want to design future sensing systems by accounting for possible attack models from first principles. %A preliminary direction is that we can check for consistency among different sensors physical layer properties. 

% Security and localization are siloed domains traditionally. However, there is an opportunity in working across the domains to make localization secure, and to leverage localization and sensing for enabling secure interaction among physically co-located devices. As a start, I visited a security research group in another university for a week and started looking at how the ToF localization stack I have designed would change for these systems to be robust to certain physical layer attacks. One promising direction is that we could use multiple sensors and check for consistency among them, since each has different physical layer properties. 

%Security and localization are siloed domains traditionally. However, there is an opportunity in working across the domains to make localization secure, and to leverage localization and sensing for enabling secure interaction among physically co-located devices. As a start, I visited a security research group in another university for a week and started looking at how the ToF localization stack I have designed would change for these systems to be robust to certain physical layer attacks. One promising direction is that we could use multiple sensors and check for consistency among them, since each has different physical layer properties. 

%, since each sensor type has differe.%the problem space with a security research group.
%After working with several localization technologies, I was interested in analyzing how the localization stack I have developed would change if we have to re-design these systems to be robust to certain types of physical layer attacks on the ranging signals.
 %I would also explore new location-based techniques for secure device discovery and interaction. % and am analyzing how the localization stack I have developed would change to be robust to certain attack models.\\

\paragraph{Next-Generation Communication and Localization. }
(re-writing this) Location awareness will be critical for efficient allocation of resources in next generation communication. 
%New challenges and opportunities emerge when sensing, location, and communication technologies of the future co-exist. For instance, l
Location awareness in fifth generation (5G) wireless networks can predict channel characteristics and connectivity. Location-aware distributed devices, such as a network of drones, can coordinate together for beamforming and creating large MIMO arrays for mmWave technology. To solve research challenges in this space, I can draw from my experience in localization and communication, and my approach of working across layers of the system stack. \\
% New challenges and opportunities emerge when sensing, location, and communication technologies of the future co-exist. For instance, location awareness in fifth generation (5G) wireless networks can enable better allocation of resources, by predicting channel characteristics and connectivity. Location-aware distributed devices, such as a network of drones, can coordinate together for beamforming and creating large MIMO arrays for mmWave technology. To solve research challenges in this space, I can draw from my experience in localization and communication, and my approach of working across layers of the system stack. \\

\paragraph{Mixed Reality. }
Mixed reality is a promising future direction to interact with future smart devices and to close the digital-physical loop in real-time. However, it requires accurate representation of both the display device and the physical objects in the same reference frame, for which traditional vision-based methods are insufficient.
%I am interested in closing the digital-physical loop with mixed reality applications. %
%I am interested in both sensing methods to convert physical content into virtual content and in real-time overlay and updates of virtual content on physical objects. 
I want to develop design principles for fusing multiple sources of information, such as vision, emerging localization technologies, and communication from overhead LEDs at the low-level to create robust mixed reality systems. As a start, I have shown how we can improve state-of-the-art mobile Augmented Reality (AR) using beacons and magnetic fields \cite{mobileAR} and I am exploring a VLC-based approach for discovering AR content. 

%\paragraph{Resource-Constrained Sensor Data Processing. }

\paragraph{New Applications and New Paradigms. }
I want to enable a future where we use minimal resources to go from smart devices to meaningful applications. 
%I will continue to tackle problems across theory and systems to design practical and reliable CPS applications. 
With large amounts of sensor data, resource constraints on embedded devices, privacy challenges, and diverse end-applications that make	 inferences from the same data, we require new paradigms for 
sensor data processing at the edge. % and sensing and system models that are suitable for edge devices. % I would explore efficient paradigms for processing sensor data at the edge, and in developing embedded machine learning algorithms for sensor data processing. 
I am interested in tackling challenges for new application domains that are moving towards `smarts' - such as healthcare, manufacturing, agriculture and training. 
I believe I can apply my approach of working across the system stack and working across theory and systems %designing embedded sensing systems, system models, design tools, and estimation algorithms to these domains where there is tremendous diversity in the sensors, devices, and application requirements. I will continue 
%to tackle problems that span theory and systems 
to tackle the challenges of designing practical and reliable cyber-physical systems. 








\footnotesize
\bibliographystyle{abbrv}

\bibliography{references}  % sigproc.bib is the name of the Bibliography in this case

\end{document}

%My future research would span several technical and application domains, and I plan to seek funding from government agencies and industry.

%\paragraph{Security. }
 %The system models assumed for sensing applications typically do not account for attacks at the physical layer. 
 %I am interested in understanding how multiple sensor feeds can make systems more robust to physical layer attacks. The intuition is that different types of sensors have different models and we can check for consistencies among them. %systems are not usable unless secure.\\
% %After working in the area of localization, I started thinking about what are the security implications in localization. 
 %As a start in this direction, I have started exploring this problem space with a security research group and am analyzing how the localization stack I have developed would change to be robust to certain attack models.\\ %More broadly, I believe unless these systems are secure 
% \paragraph{Security. }
% The future smart devices necessarily have to be secure. Sensing and security together open up two kinds of research challenges - first I am interested in designing sensing systems that are secure as well as designing novel sensing applications to establish secure device interactions. %for security applications such as detecting %With growing interaction between heterogeneous devices, we have to design for security from first principles, rather than including it as an after-though. I am interested in two complementaty probelms, the first is using sensing systems to authentical and attest devices, 
% designing system models and inference algorithms that take into account attack models. One direction understanding how multiple sensor feeds can make systems more robust to physical layer attacks. The intuition is that different types of sensors have different models and we can check for consistencies among them. 
% As a start in this direction, I have started exploring this problem space with a security research group and am analyzing how the localization stack I have developed would change to be robust to certain attack models. \\

%My future research 
%I enjoy solving problems that emerge when we look at systems holistically, and am working across domains, and am looking forward to the challenges future.
 % and I would collaborate with researchers, industry and government organizations.



