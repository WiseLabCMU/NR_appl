\documentclass[10pt]{article}

%\usepackage{fancyhdr}
 
%\pagestyle{headings}
%\markright{John Smith}

\date{}

\usepackage{amsmath}    % need for subequations
\usepackage{graphicx}   % need for figures
\usepackage{verbatim}   % useful for program listings
\usepackage{color}      % use if color is used in text
\usepackage{subfigure}  % use for side-by-side figures
\usepackage{hyperref}   % use for hypertext links, including those to external documents and URLs
\usepackage{lipsum}
\usepackage{url}

\usepackage[margin=1in]{geometry}

\usepackage{graphicx}
\usepackage{balance}
\usepackage{comment}
\usepackage{amssymb,amsmath}
\usepackage{caption}
\DeclareCaptionType{copyrightbox}
\usepackage{subfigure}
\usepackage{enumerate}
\usepackage{color}
\usepackage{titling}
%\usepackage{subcaption}
\newcommand{\figref}[1]{Figure~\ref{fig:#1}}
\newcommand{\tableref}[1]{Table~\ref{tab:#1}}

\newcommand{\compactimg}{\vspace{-12pt}}

\clubpenalty=10000 
\widowpenalty=10000
\setlength{\parindent}{0cm}



\begin{document}
\pagenumbering{gobble}

\setlength{\droptitle}{-5em}

\title{{\Large  Research Statement}}
%\vspace{-3em}
%\author{\textit{Niranjini Rajagopal}}
\maketitle

\vspace{-4em}


% What I am interested in

%I am interested in solving problems that emerge when we integrate cyber-physical systems (CPS)  
%I am interested in solving problems that emerge when physical systems in the real-world are integrated with physically distributed computing systems over a network. 
%Cyber-Physical Systems (CPS) is the paradigm 
I am interested in a principled approach for sensing and estimation in cyber-physical systems (CPS) in the real-world. Some examples of CPS applications are smart grids, internet-of-things, swarm robotics, smart infrastructure.
%CPS are systems where computing is integrated with the physical world through physically distributed sensors and actuators over a network. The goal of this integration is to monitor and change the state of the physical world.  
CPS refers to physically distributed networked embedded systems integrated with the physical real-world through sensors and actuators. The goal of this integration is to monitor and change the state of the physical world. 

Every CPS application has a system solution. The real challenge is in designing solutions that navigate the trade-off between performance and resources in a systematic manner and work in the real-world at scale. %The ch  %We care about two aspects - performance and resources. 
The performance of the embedded system we design is characterized by the accuracy, timing and reliability of the estimating and controlling the physical system. The resources are characterized by density and cost of devices, power consumed by devices, and time/effort invested in deploying or calibrating the system. My goal as a system designer is to achieve high performance with low resources. %Every CPS application has a system solution. The challenge is in designing solutions that navigate the trade-off between performance and resources in a systematic manner and work in the real-world at scale.
%These are systems where computing is integrated with the physical world through  physically distributed sensors and actuators over a network. The goal of this integration is to monitor and change the state of the physical world.  

%They rely on a model of the system.
%Advancements in low-power microcontrollers, radios, sensors, devices, communication and network technologies are now pushing CPS to a point where .\\
 

%Why is this hard? Large-scale real-world physical systems are complex, have temporal dynamics, and vary with environmental parameters that are difficult to characterize. 
%Hence, they are difficult to model in a systematic approach.
%The performance of the embedded system we design is characterized by the accuracy, timing and predicatiability of estimating and controling the physical system.
%Our goal is to achieve 
%Hence, they are difficult to model in a systematic approach.  % that are hard to , and as a result are difficult to model. 
%This makes it hard to design systems in a systematic manner to have reliable performance in the real-world. 
%This makes it hard to design systems %There are several ways to design these systems. However, it is hard to the real challenge is to design systems that can achieve reliable performance in the real-world, with contraints This makes it hard to design computing systems that achieve reliable performance at scale. 
%Performance depends on the application, and is measured by - accuracy, timing, predictability of estimating and controling the physical system.  A common approach to design systems to meet the performance requirement in the real-world is to use more resources - such as higher density or more expensive sensors. Another approach is to customize the system, such as performing environment-specific calibration to compensate for an unknown or low fidelity model of the system. 
%Every CPS application has \txtit{a} solution but the real challenge is in navigating the design space with trade-off in performance and amount of resources used. The first step to navigating this trade-off is that we need a systematic approach to quantify 
% My research aims to design approaches for CPS that are systematic
%My research understands this trade-off between performance and resources to design systems 

I work at the interface of signals and systems - both physical systems and embedded systems. 
%My work addresses this challenge through a sensing and estimation approach. I work at the interface of signals and systems - both physical systems and embedded systems. 
% I work at the intersection of theory and practice and 
%I develop abstractions that are general enough to apply principled approaches, and practical in the real-world.  
My goal is to do the best with the available sensors and resources, by (1) opportunistic use of sensors (2) designing better system models and (3) designing new estimation algorithms. I develop these through experiments and an understanding of signals and systems from first principles. %With systems, I refer to physical systems as well as embedded systems. 
% physical systems or 
More broadly, I apply tools from estimation, signal processing and optimization to embedded sensing systems. I implement and evaluate my approaches by deploying systems in the real-world. \\

\textbf{Overview of my research in indoor localization:}\\
%Location and time are key properties of the physical world. 
%My dissertation focuses on indoor localization - an instance of a CPS problem, and a fundamental property for several CPS applications. 
My dissertation focuses on indoor localization - a classic instance of a CPS problem, with physically distributed devices, real-time properties, networked components, embedded and mobile devices. Location is also a fundamental property for several CPS applications. 
%Localization tighly couples computing with the physical world.  
%with tight coupling between computing and the physical world. Location is a fundamental property for CPS applications. 
There are more than a million papers on the topic of indoor positioning and indoor localization. Yet localizing mobile devices inside buildings isn't an everyday reality. A systematic scalable approach to indoor localization that can navigate the trade-off between performance and resources is hard due to: (1) complexity of indoor environments (2) temporal dynamics due to people, devices and things moving (3) each environment is unique and varied. %As a result it is hard to model the interaction of signals with indoor spaces. This makes it hard to design system solutions that can navigate the trade-off between performance and resources. %Ideally, we want a system that is accurate all the time. 
Performance in indoor localization is characterized by accuracy, acquisition time (time to initialize), and the update rate. The resources are characterized by amount of infrastructure deployed, additional hardware on the thing/person/device being localized, and the effort and time to calibrate, map or setup the system.  \\%Existing solutions for mobile localization are of two types. 
%Localization app
Let us consider the various localization solutions for mobile devices. Infrastructure-free approaches use sensors on mobiles (inertial, magnetic, camera), signals from WiFi Access Points in the environment, and floor plan information. These systems require environment-specific and device-specific calibration. Infrastructure-based systems are capable of accurate positioning but they deploy beacons and require these beacons to be setup and mapped. Most importantly, unless there is a high density of beacons, both approaches often require the person to move some distance before the accuracy is acceptable, resulting in high acquisition time. No existing solution is accurate, instant, updates continuously, has zero-cost, is compatible with commodity devices and most importantly, works reliably at scale in the real-world. \\

%My dissertation builds upon 
%My approach to indoor loclaization is to  

%Across existing literature in localization, accuracy is achieved by 
%difficulty in modeling systems complexity of indoor spaces design trade-offs between performance and cost; requirin Some 

%My vision for an indoor localization ecosystem of the future is that devices . Is this vision possible? 
%My work addresses these challenges, towards the vision of an infrastructure-free opportunistic indoor localization ecosystem that can accurately and instantly locate things, devices and people.\\
My vision for indoor localization is  an infrastructure-free opportunistic indoor localization ecosystem that can accurately and instantly locate things, devices and people. I envision that in future, IoT devices, smart appliances, WiFi access points, and mobile devices such as phones, tablets, laptops will have ranging (time-of-flight or time-difference-of-arrival) capabilities. The ranging would be based on current technologies (ultrasonic, ultra-wideband) or emerging technologies (BLE5, 802.15.4z, WiFi 802.11mc, mmWave). These devices when stationary will act as beacons to localize mobile devices. %Developments in academia, industry and standards are pushing ranging technologies into commodity devices. Current ranging technologies include ultrasonic ranging, ultra-wideband ranging and BLE ranging. Emerging standards such as BLE5, 802.15.4z, WiFi 802.11mc are pushing ranging capabilities into commodity devices. \\


My dissertation takes steps towards this vision by solving three problems faced by beacon-based ranging systems in the real-world. First, there is no systematic approach to quantify the quality of a beacon configuration. 
%Given a limited number of beacons, where to place them indoors? 
My first contribution is a new abstraction of floor plans and beacon ranging which (1) reduces the number of beacons required as compared to state-of-art, (2) proposes a new metric that quantifies the quality of a beacon placement (3) proposes algorithms for beacon placement \cite{rajagopal2016beacon}. An important practical problem faced by range-based systems is accurate location acquisition in the presence of non-line-of-sight signals and few line-of-sight signals. My second contribution is a localization solver that leverages the floor plan geometry to acquire accurate location, in the presence of non-line-of-sight signals and low line-of-sight beacon density \cite{rajagopal2018enhancing}. Though most of the current systems focus on location, an important requirement for applications such as navigation and augmented reality is knowing the device orientation as well. My third contribution is a novel orientation acquisition approach, using beacons and magnetic field sensing \cite{mobileAR}. In addition, I have designed algorithms for tracking \cite{lazik2015alps}, beacon mapping \cite{lazik2015alps,mobileAR} and reducing the number of beacon by enabling ToF ranging from TDoA ranging \cite{rtas-alps-platform}\\

%To achieve my vision for indoor localization, 
I contribute to fundamental research, implement and deploy systems and collaborate with external organizations. \\%will continue to work with industry and standards organizations, in addition to contributing to fundamental research. I implement and deploy systems in the real-world.\\
\textbf{Demos:} I have demonstrated my solutions with an ultrasonic beacon platform that localizes unmodified phones \cite{lazik2015alps, lazik2015alpsdemo, rtas-alps-platform}. 
I implemented an early version of the floor-plan aware solver  \cite{rajagopal2018enhancing} in our system that won the Microsoft Indoor Localization Competition in 2015. Our orientation acquisition technique applied to mobile augmented reality won best demo at IPSN 2018 \cite{rajagopal2018welcome}. Our localization algorithm designed for accurate tracking \cite{mobileAR} won the Microsoft Indoor Localization Competition in 2018 with Ultra-Wideband beacons.\\
\textbf{Industry:} Our ultrasonic localization platform is spun out into a start-up. We have demonstrated mobile augmented reality applications with this system. I worked with Apple at a summer internship on one of their earliest prototypes of WiFi ToF localization, which eventually resulted in product impact. I am currently working with Texas Instruments on BLE ToF and implementing localization and SLAM on BLE. \\
\textbf{Standards:} I am working with National Institute of Standards and Technologies on firefighter localization in emergency scenarios. We are working on an infrastructure-free localization paradigm I believe that for any infrastructure to become part of buildings (such as fire sprinklers or exit signs), standards enforced for safety can be a key enabler. We are exploring the possibility of the exit signs as locations for low-power beacons to be deployed in emergency scenarios.\\



% I demonstrate my approaches by implementing them and deploying systems. Our ultrasonic-based localization system  \\
% \cite{rajagopal2016beacon}\\
% \cite{rajagopal2018enhancing}\\
% \cite{lazik2015alps}\\
% \cite{lazik2015alpsdemo} : demo\\
% \cite{} : AR demo

%Time-of-flight ranging 
%My dissertation addresses important challenges towards this vision, in a principled manner.  
%Range-based 
%I take an alternate My disseration takes an alternative approach. I 
%ToF-based localization accuracy is 
% quantifiable precisely by the time-bandwidth product of the ranging channel, and the relative location of infrastructure with respect to the . The accuracy of positioning is limited by the ranging accuracy and the relative location of beacons. There are engineering challenges for why ToF ranging is not part of commodity devices. But there are emerging standards for ToF ranging 
% ToF-based localization accuracy is quantified precisely. %are capable of accurate positioning. The limits can be quantified precisely. 

% Indoor localization as a problem space requires wo
% emerging ToF technologies - envision future ecosystem.\\
% Contributions span fundmantal and applied.\\
% On fundamental - no systematic approach - IPIN, computational geometry
% IPSN
% Orientation
% SLAM
% Ultrasonic localization - cite papers, demo - startup - won competition.\\
% Most precise phone-based localization system.\\
% RF - Apple - earliest prorotype.\\
% TI - BLE SLAM
% UWB - NIST, won competition.\\


\section{Indoor Localization}
I now describe in detail my main contributions to indoor localization. %I have adopted an opportunistic sensor-fusion approach.\\

% blah blah blah...\\
% blah blah blah...\\
% blah blah blah...\\
% blah blah blah...\\
% blah blah blah...\\
% The first challenge for localization is that it has to support heterogeneous devices with varied sensing capabilities. My approach to this is to
% %I believe that to enable indoor localization to become a reality, 
%  embrace the emerging ranging and localization technologies that are finding their way into newer standards and mobile devices, and build an ecosystem that enables these technologies to scale up and be integrated with heterogeneous sensors. \textit{talk about range-based localization systems and envisioned ecosystem. }.\\%solve the problems that these systems face which scaling up. My world adopts a sensor fusion approach 
% %My approach: range-based localization\\

The first problem is that there is no systematic understanding of how and where to place range-based beacons. Currently, beacon placement is done empirically by domain-experts, and indoor spaces are over-provisioned with beacons. When deploying beacons, an installer faces the conflicting objectives of reducing the number of beacons to be placed and increasing system coverage, accuracy and resilience. Existing approaches for trilateration require three or more beacons to determine a unique position solution. First, we show that with prior knowledge of the map and a model of beacon coverage, it is possible to uniquely localize with only two beacons. This not only reduces installation cost by requiring fewer nodes, but can also improve robustness. One of the main challenges with respect to beacon placement algorithms is defining a metric for estimating performance. We propose augmenting the commonly used Geometric Dilution of Precision (GDOP) metric to account for indoor spaces, and define a new metric - Indoor GDOP. We adopt the GDOP since the Cramer Rao Lower bound on the variance of the location estimate is proportional to the GDOP and the ranging error, under the assumption that the ranging noise is additive white noise Gaussian. We then use this indoor GDOP metric and propose new beacon placement algorithms that optimize for  coverage and expected accuracy. We designed a toolchain for the beacon placement algorithms on floor plans. When applied to a set of real floor plans, our placement algorithms are able to reduce the number of beacons between 22\% and 60\% (33\% on an average) as compared to standard trilateration. I further extended this work in collaboration with Prof. Jie Gao from Stony Brook and her student. We mathematically formulated the beacon placement algorithm and proposed placement algorithms with provable guarantees \cite{beaconplacementtheory}. I believe that such systematic approches and automated tools for beacon placement are necessary while scaling up these systems from labs to building-scale. \\


The second problem is that in reality, signals from line-of-sight beacons get blocked, we also receive incorrect non-line-of-sight (NLOS) signals. To maintain accuracy despite these, either beacons are over-provisioned (trading-off cost) or other sensor measurements are used over time, or the user is asked to walk around (trading-off time-to-estimate). Hence, none of the existing approaches can acquire location with low beacon density. To solve this problem, I design a new floor-plan aware location estimation technique that can instantly acquire location  
%Our second contribution is a localization algorithm that is able to instantly acquire location 
in the presence of low-beacon density and incorrect non-line-of-sight (NLOS) signals. 
%To get accurate location in the presence of non-line-of-signals, existing approaches either deploy beacons with high-density (increasing cost) or use inertial-tracking to converge on an estimate over time (increasing time-to-first-fix). 
This solving approach uses the floor plan information and can disambiguate multiple feasible locations taking into account a mix of LOS and NLOS hypotheses to accurately localize instantly with significantly fewer beacons. We demonstrate our geometry-aware solving approach using a new ultrasonic beacon platform that is able to perform direct time-of-flight ranges on commodity smartphones. The platform uses Bluetooth Low Energy (BLE) for time synchronization and ultrasound for measuring propagation distance. When evaluated in multiple real deployments, our approach was able to improve the 80\% localization accuracy from 4-8m to 1m in low-beacon density and NLOS conditions. We are able to detect and remove NLOS signals with 91.5\% accuracy.\\

The third problem I addressed was orientation acquisition on mobile devices. While beacons can provide accurate localization, orientation acquisition is still a challenge. Orientation is necessary for applications such as mobile augmented reality. Most existing systems estimate the orientation by requiring the user to walk around for some distance, or relying on visual features in the environment.  %that is required for mobile Augmented Reality still remains a challenge. 
We show how the combination of Visual Inertial Odometry and beacons can be used to crowd-source a dense magnetic field map. Future users can use this map at startup to calibrate their compass in order to instantly estimate orientation. We show in a demonstration system running on an iPhone that can acquire location with 80\% percentile 3D accuracy of 24cm. We also see that our magnetic field mapping approach can instantly estimate orientation with 80\% percentile accuracy of 16 degrees.  We demonstrate an end-to-end system that automatically configures beacons, generates the magnetic field map and supports a mobile AR applications that works in any environment without requiring the sharing of large and often fragile point-cloud maps. \\

Finally, I have designed algorithms for tracking \cite{lazik2015alps}, beacon mapping \cite{lazik2015alps,mobileAR} and reducing the number of beacon by enabling ToF ranging from TDoA ranging \cite{rtas-alps-platform}.\\

%Applications such as augmented reality require the device's orientation in addition to location. 
%Improving Mobile Augmented Reality Re-localization \\ using Beacons and Magnetic-field Maps

%The third problem is that for many applications, we require to know the orientation of the device in addition to location. However, 
%Our third contribution is to estimate instant orientation, in addition to location. 
%typical approaches for orientation acquisition use the mobility of the device (trading-off time-to-estimate). I propose an approach where we can leverage the magnetic field along with the localization system to 
%We use the location estimated from a beacon infrastructure along with a map of magnetic field direction to 
%instantly estimate the device’s orientation immediately without requiring the user to walk around. Our system provides 90\% localization accuracy of 24cm with LOS beacons and 90\% orientation acquisition accuracy of 16° from magnetic field sensing. blah blah blah...\\
%blah blah blah...\\
%blah blah blah...\\
%blah blah blah...\\
%blah blah blah...\\
% blah blah blah...\\
% blah blah blah...\\


%My approach to this is to abstract the floor plan and beacon ranges to models where we can apply concepts from estimation theory. Currently, beacon placement is done empirically by domain-experts, and indoor spaces are over-provisioned with beacons. Existing approaches require three or more beacons to determine a unique position solution. I show that with prior knowledge of the map and a model of beacon coverage, it is possible to uniquely localize with only two beacons. I introduce a metric to quantify the quality of a beacon placement and design an algorithm for systematically placing beacons in a floor plan. This is based on the Cramer Rao Lower bound on the location estimate, under the assumption that the ranging noise is additive white noise Gaussian. When applied to a set of real floor plans, this placement algorithm is able to reduce the number of beacons by 33\% on an average as compared to standard approaches. \\

%we address the problem of range- based beacon placement given a floor plan to support indoor localization systems. Existing approaches for trilateration require three or more beacons to determine a unique position solution. We show that with prior knowledge of the map and a model of beacon coverage, it is possible to uniquely localize with only two beacons. This not only reduces installation cost by requiring fewer nodes, but can also improve robustness. One of the main challenges with respect to beacon placement algorithms is defining a metric for estimating performance. We propose augmenting the commonly used Geometric Dilution of Precision (GDOP) metric to account for indoor spaces. We then use this enhanced GDOP metric as part of a toolchain to compare various beacon placement algorithms in terms of coverage and expected accuracy. When applied to a set of real floor plans, our approach is able to reduce the number of beacons between 22% and 60% (33% on an average) as compared to standard trilateration.


% The final problem - is mapping. We also present a pedestrian-aided automatic mapping approach for mapping the beacons and magnetic field rapidly upon setup... \\
% blah blah blah...\\
% blah blah blah...\\
% blah blah blah...\\
% blah blah blah...\\
% blah blah blah...\\
% blah blah blah...\\

%In summary, there are several ways of solving the localization problem. However, the real challenge is in solving it in a manner that maintains performance (high accuracy, low time-to-estimate) in a manner that is also cost efficient (low infrastructure, mapping and setup effort). My work addresses these challenges in a principled manner by creating new models and estimation techniques. \\


\section{My research in other areas in CPS}
In addition to location, time is a fundamental property of CPS. I have worked on problems in time-synchronization \cite{buevich2013hardware, dongare2017pulsar, rtas-alps-platform}.  I have also worked on CPS applications of energy metering and visible light communication, using a sensing approach. The challenge across these applications is in system design that achieves performance with limited resources. \\
%is in system design  I have worked on the challenges I have solved are to address the CPS design space with trade-off between performance and resources. \\
%I have worked on problems in time-synchronization in CPS, and in CPS applications of energy metering and visible light communication.  
\textbf{Energy metering}:\\
Prior to joining CMU, I worked for two years in industry on designing embedded energy metering products. % that monitor aggregated energy from a group of houses/industries. 
Though we can measure aggregated energy, it is hard to dis-aggregate individual appliances without using plug-in meters for each appliance. At CMU, I explored the problem of dis-aggregating individual appliances within a home. 
To solve this, we proposed a solution with novel sensing, a simple model and an estimation algorithm. We designed contactless battery-operated electromagentic field (EMF) sensors that we deployed near each appliance. This sensor detected appliance state transitions based on magnetic and electric field fluctuations. We modeled each appliance as a two-state device and designed an algorithm for load-disaggegation using the data from magnetic field sensors and whole house energy meter \cite{rajagopal2013magnetic} and demonstrated the system \cite{rajagopal2013demo}. 
%including reference energy meters used by utilities for calibrating individual meters, and multi-function transducers deployed in substations to monitor aggregated energy from a group of houses/industries. 
%However, we did not have ways to monitor energy within homes at appliance-level. I explored this problem when I started at CMU. The challenge was in solving this problem to achieve high performance (accurate dis-aggregation of load) and use low resources (non-intrusive monitor appliances). 
%detecting appliance state-transition using magnetic field and correlating the sensor information with whole house energy meter data to dis-aggregate individual appliances . 
Working with energy metering and the electrical infrastructure of homes, a question that emerged was can we opportunistically use the power lines as a communication channel indoors? %Though there were standards for power line communication (PLC), the interesting question there was is it possible to design a communication scheme for PLC that could use the WiFi communication stack? 
To explore this question, I interned at Texas Instruments Communications and Systems Lab and designed a power line communication-WiFi hybrid communication scheme in simulation at the MAC and PHY layer.\\

\textbf{Visible Light Communication}:\\ 

The next question that emerged was, even if the PLC could be used as a backbone infrastructure, how can we communicate data from the electrical infrastructure to mobile devices? This question was timely, as back in 2013, LED lights were starting to phase out incandescent lights, and it was clear that LED lights will become pervasive in the future indoor infrastructure. LED lights turn on and off at a high frequency. 
%I then explored the question - can we opportunistically use the lighting infrastructure indoors to send data to phones? 
%The question we asked is - can we use light as a medium and modulate the signals on LED lights to send data, and sense this data from phones? 
%The motivating application for this was to coarsely localize phones indoors based on which lights are in proximity, and to leverage lights as a new communication infrastructure indoors. 
The challenge was that for the lights to be flicker-free, they have to be modulated at a frequency much higher than the camera frame rate. To communicate data despite this limitation, I designed a novel sensing approach that exploited the rolling-shutter effect of cameras on phones to detect a high frequency light signal. I extended this to a modulations scheme that supported multiple lights. The main challenge was in maintaining performance when the user holds the phone normally, rather than pointing at the light. To overcome this, I modeled the exposure and focus control of the camera as filters that respond differently to the light signal and the scene captured and used this to improve the signal-to-noise ratio. This system was one of the earliest systems to show that we can send data from overhead LED lights to phones using the rolling shutter effect \cite{rajagopal2014visual}.  We demonstrated this system \cite{rajagopal2014demonstration} and also extended the work \cite{rajagopal2014hybrid} by designing a hybrid communication scheme where a single light transmits two independent data streams on a single channel using a combination of modulating the duty cycle and frequency, to communicate simultaneously to a phone and a photo-diode. This hybrid system was presented in the first Visible Light Communication (VLC) workshop. Subsequently, over the past four years the field of VLC has grown significantly with several rolling-shutter based approaches for communication and localization, in both academia and start-ups. Our work was on of the earliest works in using VLC for localization.

% \section{My research in other areas in CPS (short version)}
% In addition to location, time is a fundamental property of CPS. I have worked on problems in time-synchronization \cite{buevich2013hardware, dongare2017pulsar, rtas-alps-platform}. Another CPS application I have worked on is electricity sensing in buildings and smart-grid scenarios.  Understanding appliance-level electricity consumption in a building is insightful, but it is impractical to install plug-through power meters for appliances with inaccessible wires and outlets.  We built a system that estimates the energy consumption of individual appliances using a wireless network of contactless electromagnetic field sensors deployed near each appliance, and a whole-house power meter \cite{rajagopal2013magnetic, rajagopal2013demo}.  Another area of my focus has been in the area of communications where we built one of the first prototypes of a system able to transmit data from LED lights to unmodified smart phones.  We exploited the rolling shutter camera sensors on these devices to detect high-frequency changes in light intensity \cite{rajagopal2014visual, rajagopal2014demonstration} and extended the system for low-power tags \cite{rajagopal2014hybrid}.  At a summer internship at Texas Instruments, I extended this work by exploring the design of hybrid Power Line and WiFi communication systems.


\section{Future Work}
My longer term vision is to build systems that 
around sensing, perception and mobility. The complexity of the real-world signals and systems makes this space challenging and interesting. I am looking forward to collaborating with industry and researchers in diverse domains to solve problems in this space. 

%CPS have high impact but it is hard to design systems that perform well in the real-world with limited resources. %I enjoy working at the interface of signals and systems to solve these problem in this space.
% challenges. Working at the interface of the physical systems and embedded systems, and designing CPS to navigate the trade-off between performance and resources introduces hard research problems with high impact in the real-world. 
%My future research direction includes localization and broader CPS challenges and applications. 
%achieving the vision of infrastructure-free indoor localization, enabling next generation applications, solving 

%Research plan includes extending localization to new applications, as well as applying my approaches to domains beyond localization.\\
%\textit{Want to work on problems where physical world/ physical processes are tightly coupled with computing ; high performance with low resources --> high impact and hard problems.}\\
\subsection{Indoor Localization}
I want to realize the vision of an opportunistic infrastructure-free indoor localization ecosystem. My dissertation has taken steps towards this vision but there are still several problems to be solved. 
On the estimation side, we can draw from the theory on mobile network localization. The challenges are in continuously mapping devices as well as localizing devices when the network is sparse and inertial measurements are poor. %I can draw upon ideas in location acquisition with low-beacon density and sensor fusion to address these challenges. 
We also require novel discovery and medium access mechanisms for such an ecosystem since (1) devices appear, move and disappear, (2)  mixed-media ranging technologies such as BLE, WiFi, UWB, ultrasonic co-exist (3) beacons have to be continuously discovered and devices switch between acting as beacons and requesting for localization service. Most existing work is focused on accuracy and assume that the beacons are fixed, known and powered. %We require novel discovery and medium access mechanisms to solve these problems. 
As steps towards these problems, I am working with NIST on firefighter localization using a networked mobile localization approach. Further challenges includes distributed sensor fusion at the edge. Another area of interest in network localization is using a network of devices for beam-forming in emerging communication technologies. %Location of the devices is key here.
% Both in emergency and in general scenario.\\
% NIST firefighter localization.\\
% Mobile network localization\\
% IMU accuracy\\
% Crowd-sourcing 
In summary, the challenges span sensing, estimation, medium access, communication and mapping. % and I would build on ideas from sensor-fusion and modeling to solve these challenges.
%In all the CPS applications I have built, the scheduling and MAC problems have been different.
%On the mapping side, we require crowd-sourced approaches to build maps of floor plans, beacons and other environmental signatures. The challenge is in building and sharing these maps 

%For smart devices indoors to act as beacons when stationary 

\subsection{Enabling next generation applications: Mixed Reality}
Mixed reality is an exciting application that enables users to interact in new ways with the physical world. It has impact in domains ranging from entertainent, manufacturing, industrial, design, medical and training. State-of-art systems today 

Mixed reality is an exciting CPS application with new challenges due to tight coupling between the digital and physical world. Mixed reality has applications is diverse domains ranging from manufacturing, entertainment, medical and training. Location is fundamental to mixed reality. For mixed reality to become a reality, we require accurate continuous estimation of the display device's pose and all physical and virtual entities that are part of the user experience to be represented in the same reference frame. I applied the location and orientation acquisition techniques for persistent mobile augmented reality that enabled multi-user applications and virtual content to persist in the same physical location over time. To interact with objects that do not have localizing capabilities we require new approaches. As part of the CONIX research center, I am working with other students on an LED-based visual marker for mixed reality applications. In future, I want to collaborate with researchers in computer vision to integrate visual information with other sensing signals for markerless mixed reality. 

% An ideal user experience would require (1) accurate, continuous estimation of the display's pose with respect to the physical world, and (2) estimating the pose of other devices 


% Mixed reality as an application takes CPS challenges to the next level. In mixed reality, the digital and physical worlds are tightly coupled. 
% However there are other challenges. I applied our indoor localization work for mobile augmented reality. We are now integrating it with AR widgets. The next challenges are in markerless localization of things and devices that are not capable of ranging sensors. Some preliminary ideas are that,\\
% Communication and localization

% MArkerless AR 
% Integration with vision
% Physical search of devices
% Time-sync and latency --

% Applications are diverse - education, healthcare. Because human perception is in the loop, performance in terms of accuracy and latency is critical. Looking ahead, one direction I am interested in pursuing is in quantifying the performance of mixed reality systems in terms of the final application. For instance, when mixed reality is used for training and education, can we quantify the effect of the mixed reality experience on the user's training or improvement in learning? Some preliminary ideas are that 

% A concrete use case is that..\\
% Applications in 

% I believe mixed reality pushes CPS challenges to the limit\\
% Digital and physical worlds are tightly coupled in this application.\\
% Because users are in the loop and humans perception is in the loop - performance (accuracy and latency) is critical \\
% Challenges -  in estimation, timing and communication\\
% Location is a key primitive to MR.\\
% Mixed reality has applications in domains such as healthcare, education, manufacturing, entertainment. \\
% I would like to collaborate with researchers in robotics to work on problems related to robot

\subsection{Redesigning sensing systems with security as a performance metric}
In addition to accuracy, timing and reliability, I want to design systems where security is performance metric that we account for. CPS are prone to attacks across various layers of the system stack - signals, sensors, network, and data processing. I am interested in the problems that arise due to attacks at the signal or physical layer. This problem space is challenging since the real-world is hard to model and the abstractions made by the data processing algorithms are likely not robust to attacks. For the system to be robust to attacks, we have to utilize additional resources. %main challenge I foresee is that the models used for estimation do not account for attacks. 
%As a result there is a trade-off between making systems secure and utilizing lower amount of resources. %security as a performance metric trades-off with higher amount of resources . 
As first steps in this direction, I want to explore problems in context of range-based localization. At the physical layer, there are two types of attack possible on the range/distance measurement - distance enlargement and distance reduction attack. The first question is - can we detect attacks on signals from a beacon? Some preliminary ideas are that we can use additional sources of information  such as the floor plan and check for consistency among signals from multiple beacons, similar to our approach for localizing in the presence of non-line-of-sight signals. This raises another challenge - how can we distinguish an attack from a real-world outlier? One possible direction is to use other sensing sources of information such as inertial sensors, as each sensing modality will have a different attack model that has to be satisfied for the attack to be undetectable. %I can build upon our multi-sensor approach and check for consistencies between other sources of information such as inertial signals. 
Other questions I am interested in is - how would the beacon placement change, and how much more beacons do we require if we want security guarantees? Given robustness to certain attack models at the physical layer, how do the higher layers such as the MAC layer and device discovery process change? %Imagine a group of users creating AR content tied to physical locations that are not to be public but In a mixed-media ecosystem with several location-providing beacons and devices, and various applications we have to start thinking about whether access control and %The third question is, based on what attack modesl the physical layer is robust to, how would the MAC layer change, and what mechanisms can we introduce? 
This summer, I briefly visited Prof. Sdrjan Capkun's Systems Security group at ETH Zurich to start exploring some of these problems in the area of secure localization. In the longer term, I am interested in a principled approach to sensing and estimation with security as a performance metric. %This space is challenging since the real-world is hard to model and the estimation algorithms use abstractions that are likely not robust to attacks. For the system to be robust to attacks, we have to utilize additional resources. % for attacks, we have to use additional resources.  %I have started exploring these questions with 
%GPS is prone to attacks, but indoor spaces are private unless outdoors spaces. 
%Accounting for security raises more questions, such as in the future should an infrastructure be authemticating a device, i scenarious where users hae access t different physcal resources, or to differtn digigtl reoures (AR conetnet). Another paradigm, is that users authenticate which infrastructure theytrust in a mixed ecosystem with several ranging technologies and devices owned by different users. 

% Another problems space I am interested in is security in sensing systems, at the physical layer. For instance, localization systems are not designed to be robust to security attacks. Security is not part of the performance specifications while researchers are advancing this space. In range-based localization, some of the problems are - can a certain placement of beacons be more secure? While this problem has been studied for single room deployments, this is still an open problem for large -scale deployments. One of the main challenges I foresee is that, it is hard to distinguish between an attack, and an outlier measurement caused by the complexity of the physical world. For instance, how can we distinguish between a NLOS signal and a distance enlargement attack? To explore some of the problems in this space, I visited Prof. Srdjan Capkun's group at ETH Zurich to understand how the localization stack would change if we account for security. I see this problem space going beyond localization, more broadly to sensing systems. I believe this is an important problem space since attacks at the physical sensing layer cannot be detected at the higher layers of the application stack. Some preliminary ideas are that we can use multiple different sensors in an intelligent manner since the attack models are different for various sensing modalities.
\subsection{Human in the loop CPS}
A human in the loop of a CPS system poses new challenges with respect to prediction and estimation, but introduces new ways to improve performance using control and action by the human. In context of localization and mapping, the first step in this direction I want to explore is human-in-the-loop beacon mapping. Can we guide the user to walk towards a certain region or walk in a certain direction to improve the mapping accuracy with reduced time-to-estimate? A related problem is can the user move the mobile device in a certain manner to reduce the time taken to discover beacons as well as the time to acquire location and orientation with improved accuracy? The idea is that mobility increases sensory input. In the longer term, I am interested in exploring problems that arise when humans interact with CPS in applications such as mixed reality for training. Here the challenge is that the performance of the CPS has to meet the human perception requirements, and the CPS has to be robust to the unpredictability of humans. 
% and we can leverage this to infer what mobility types  In the longer term, I am interested in
% Problems: mapping. Exploring a space to increase sensory input based on mobility and motion. A concrete first problem is : SLAM - seen that we require diversity in walking pattern - how can we leverage this diversity? Mapping - automated with human input - from an estimation point-of-view, the challenge is in weighing uncertainity in the human input. \\
% Systems that interact with humans - location is important\\
% Another problem I am interested in is how we quantify performance of these systems with respect to 
% The challenge in CPS design to achieve high performance with low resources is amplified when the human 
% The eventual goal of CPS are to enable applications that 
%\subsection{Broader CPS applications}
%I would like to explore applications where the challenge in sensing and estimation is novel sensing systems for diverse challenging applications such as non-intrusive medical diagnosis,  sensing for agriculture,  industrial machine health diagnosis and civil infrastructure sensing, where I believe there are rich signals from which we can make meaningful inferences that guide action. The challenge in these applications is achieving performance with limited resources.\\

%\subsection{Long-term vision}
%My longer term vision is to understand and build systems around sensing, perception and mobility. The complexity of the real-world signals and systems makes this space challenging and interesting. I am looking forward to collaborating with industry and researchers in diverse domains to solve problems in this space. 
%build cyber-phsystems that enhance
%The real-world is full of signals rich in information and the challenge is in converting signals to information, with limited resources. 
%Seamless transition between physical and cyber world. 
%Perception and sensing systems that are scalable and high performance and work with the unpredictability and complexity of the physical world and humans. Collaborate with industry, standards, researchers in other disciplines. 



%\subsection{Communication}


%Prior work I used a sensing approach 
%I worked on during my research at CMU. To solve this problem, 
%Approaches for appliance-level load monitoring included 



%While working with VLC, one of the questions that emerged was can we use power-line communication (PLC) as a backbone for VLC? 
 
% One of the earliest problems I worked on at CMU was on solving the problem of dis-aggregating individual loads from a whole house energy meter. Prior work had solved this problem by adding intrusive plug-in meters for individual loads or performing device and installation-specific calibration on the whole house meter. 
% %The question we asked is - Can we dis-aggregate loads in a non-intrusive manner without requiring
% In contrast, we solved this problem using a novel sensing approach and a generalized model for home appliances. We designed a wireless sensor network consisting of contactless battery-operated electromagnetic field (EMF) sensors deployed near each appliance which detected appliance state transitions based on magnetic and electric field fluctuations. The model we assumed was that each appliance has two states on and off. We designed an estimation algorithm that used the EMF state transitions along with the whole-house power meter data to detect appliance state transitions \cite{rajagopal2013magnetic} and dis-aggregate individual load energy consumption, and also demonstrated this system \cite{rajagopal2013demo}. Working with energy metering and the electrical infrastructure of homes, a question that emerged was can we use the power lines as a communication channel indoors? Though there were standards for power line communication (PLC), the interesting question there was is it possible to design a communication scheme for PLC that could use the WiFi communication stack?  To explore this question, I interned at Texas Instruments Communications and Systems Lab and designed a hybrid communication scheme in simulation at the MAC and PHY layer. 

% T
% Prior, worked on energy metering, NILM, demo \cite{rajagopal2013magnetic, rajagopal2013demo}\\
% Abstracted appliance state to on-off --> based on correlating mag fld to energy metering data\\
% EMF for time-sync
% TI - PLC-WiFi. Then looked at lights as an end-point for data tx
% VLC - built one of the earliest works systems. Abstracted channel between light and camera. Hybrid system. Demoed. 
% \cite{rajagopal2014visual} and extended the system for low-power tags \cite{rajagopal2014hybrid}. \\
% \cite{rajagopal2014demonstration}\\


% \section{Future research plans}
% %location - a fundamental property of the physical world, and key to CPS that ißnte.  
% CPS have immense potential, but also challenges.\\
% Research plan includes extending localization to new applications, as well as applying my approaches to domains beyond localization.\\
% \textit{Want to work on problems where physical world/ physical processes are tightly coupled with computing --> high impact and hard problems.}\\

% \textbf{Mixed Reality}\\
% I believe mixed reality pushes CPS challenges to the limit\\
% Digital and physical worlds are tightly coupled in this application.\\
% Because users are in the loop and humans perception is in the loop - performance (accuracy and latency) is critical \\
% Challenges -  in estimation, timing and communication\\
% Location is a key primitive to MR.\\

% \textbf{Heterogeneous robotic systems}\\
% We are at the point where robots are no longer confined to labs.\\
% Aerial and ground robots.\\
% Group of robots : example a drone + self driving car + mobile device coordinating and jointly estimating or doing a task.\\

% \textbf{IoT}
% Isn't IoT part of CPS? Not clear if I should keep this.

%\textbf{Broader vision}\
%Perception and sensing systems that are scalable and high performance and work with the unpredictability and complexity of the physical world and humans.



% CPS end-end design is application specific. 
% CPS - application-specific. I have worked on applications, on communication; 
% VLC - imapct.\\
% Indoor localization.\\





%As a result, we either have either theory that is geneasystems that work well in lab environments and do not scale, or theory  %This makes it hard to design CPS in a principled manner. %A principled approach to sensing and estimation in CPS\\
%My goal is to develop principled approaches that all

%Physical things move - introducing complexity\\
%dynamics of system over time\\
%High fidelity model accurately represents the system\\
%Physical + computing + networking





%My approach is to bridge the gap between theory and practice of sensing and estimation in CPS by working at the intersection of estimation, signal processing, optimization and embedded sensing systems. 
%Performance can be traded-off with cost by 
%complexity and unpredictabiity of physical systems in the real-world makes it hard for these computing systems to be reliable at scale. 

%Cyber-Physical Systems 

%I am interested in a principled approach for sensing and estimation in cyber-physical systems (CPS) in the real-world. \\
%I am interested in a principled approach for sensing and estimation in cyber-physical systems (CPS) in the real-world. 
% Why CPS - goal of CPS
% examples of CPS - swarm of robots, networked sys for indoor loc, smart grid, network of medical devices,
%CPS enable us to estimate the state of the physical world, and 
%% The physical world is complex. The ultimate goal is to accurately know the state of the physical world, and have the ability to 
%Examples of what CPS can do potentially.
%For instance, a network of drones coordinating on a task, devices estimating the state of a smart grid, personal and wearable devices together estimating health of a person, self-driving cars, etc.
%What are CPS - should be clear that there is the physical world and based on the state of the physical world we want to take some actions.
% CPS consist of sensors and actuators that interact with the physical systems, and computation components, often connected over a network.
% % What does it even mean to have a good design of CPS?
% The goal of a CPS is to monitor and effect a change in the state of physical processes. However, the complexity of physical processes pose several challenges to CPS design.
% %Given a problem, how do we quantify what is a good CPS design to solve this problem? Solutions can vary in performance (accuracy, predictability, latency) and cost incurred (computational and infrastructure resources). 
% %Challenges for CPS desigh emerge due to 
% %What are the challenges in designing CPS?
% %The real-world is complex and unpredictable, posing several challenges in the design of CPS. 
% %The complexity of real physical systems pose several challenges to CPS design. 
% First, it is hard to have accurate system models for physical systems that are part of CPS. Second, there is a trade-off between cost and performance in terms of the type, quality and quantity of sensors used.  Third, timing plays a role in both estimation and actuation, but measuring time across distributed physical systems is hard. Fourth, these systems often rely on communication and networking capabilities. 
% My work spans the problems of timing and communication, but focuses on the problems related to estimation and sensing. 
% \textit{A principled approach to CPS design is one that generalizes and also works in the real world.} 
% A principled approach to CPS design is hard since solutions can vary in performance (accuracy, predictability, latency) and cost incurred (computational and infrastructure resources). 
% %What it means to have a principled approach - this is critical
% %Why we require a principled approach - this is critical
% %Why is this hard?

% My approach to solving these problems is to work at the intersection of theory and systems of CPS, with contributions to fundamental and applied research. I apply tools from estimation theory, statistical signal processing and optimization and build and deploy systems in the real-world. 


% My dissertation is focused on location, a fundamental property of the physical world. %, key to indoor CPS.
% \section{Research in indoor localization}
% % why indoor localizatoin is important 
% % why it is hard 
% % our approach
% % Three problems
% % 	No systematic beacon placement - GDOP, CRLB. Toolchain, what we achieved. 
% % 	Instant loc acquisition
% % 	Ins orientation acq
% % 	Mapping
% % 	Applications
% % 		FF
% % 		mobile AR
% Knowing where people and things are inside buildings will impact several domains and open up new applications. Some examples are location-based physical search for things, spatially-aware communication, drones navigation indoors, and interactive mixed reality. Though several techniques exist, indoor localization at scale is not yet a reality. The challenges draw from the traditional challenges of CPS - in sensing and estimation, time-synchronization and communication. My work addresses these challenges, towards the vision of an infrastructure-free opportunistic indoor localization ecosystem that can accurately and instantly locate things, devices and people.\\ %Towards this vision, my work addresses four challenges faced by localization, which arise due to  

%\textbf{Research}
% The first challenge for localization is that it has to support heterogeneous devices with varied sensing capabilities. My approach to this is to
% %I believe that to enable indoor localization to become a reality, 
%  embrace the emerging ranging and localization technologies that are finding their way into newer standards and mobile devices, and build an ecosystem that enables these technologies to scale up and be integrated with heterogeneous sensors. \textit{talk about range-based localization systems and envisioned ecosystem. }.\\%solve the problems that these systems face which scaling up. My world adopts a sensor fusion approach 
% %My approach: range-based localization\\

% The second challenge is that there is no systematic understanding of how and where to place range-based beacons. My approach to this is to abstract the floor plan and beacon ranges to models where we can apply concepts from estimation theory. Currently, beacon placement is done empirically by domain-experts, and indoor spaces are over-provisioned with beacons. Existing approaches require three or more beacons to determine a unique position solution. I show that with prior knowledge of the map and a model of beacon coverage, it is possible to uniquely localize with only two beacons. I introduce a metric to quantify the quality of a beacon placement and design an algorithm for systematically placing beacons in a floor plan. This is based on the Cramer Rao Lower bound on the location estimate, under the assumption that the ranging noise is additive white noise Gaussian. When applied to a set of real floor plans, this placement algorithm is able to reduce the number of beacons by 33\% on an average as compared to standard approaches. \\

% The third challenge is that in reality, signals from line-of-sight beacons get blocked, we also receive incorrect non-line-of-sight (NLOS) signals. To maintain accuracy despite these, either beacons are over-provisioned (trading-off cost) or other sensor measurements are used over time, or the user is asked to walk around (trading-off time-to-estimate). To solve this problem, I design a new floor-plan aware location estimation technique that can instantly acquire location  
% %Our second contribution is a localization algorithm that is able to instantly acquire location 
% in the presence of low-beacon density and incorrect non-line-of-sight (NLOS) signals. 
% %To get accurate location in the presence of non-line-of-signals, existing approaches either deploy beacons with high-density (increasing cost) or use inertial-tracking to converge on an estimate over time (increasing time-to-first-fix). 
% This solving approach uses the floor plan information and can disambiguate multiple feasible locations taking into account a mix of LOS and NLOS hypotheses to accurately localize instantly with significantly fewer beacons. When evaluated in multiple real deployments, our approach was able to improve the 80\% localization accuracy from 4-8m to 1m in low-beacon density and NLOS conditions.

% The fourth challenge is that for many applications, we require to know the orientation of the device in addition to location. However, 
% %Our third contribution is to estimate instant orientation, in addition to location. 
% typical approaches for orientation acquisition use the mobility of the device (trading-off time-to-estimate). I propose an approach where we can leverage the magnetic field along with the localization system to 
% %We use the location estimated from a beacon infrastructure along with a map of magnetic field direction to 
% instantly estimate the device’s orientation immediately without requiring the user to walk around. Our system provides 90\% localization accuracy of 24cm with LOS beacons and 90\% orientation acquisition accuracy of 16° from magnetic field sensing.\\

% The final challenge - is mapping. We also present a pedestrian-aided automatic mapping approach for mapping the beacons and magnetic field rapidly upon setup... \\

% In summary, there are several ways of solving the localization problem. However, the real challenge is in solving it in a manner that maintains performance (high accuracy, low time-to-estimate) in a manner that is also cost efficient (low infrastructure, mapping and setup effort). My work addresses these challenges in a principled manner by creating new models and estimation techniques. \\

% \textbf{Demonstration of approach, and applications}\\
% Here talk about - above is contribution to fundamental research.\\
% But we have also demonstrated and deployed\\
% Localization competitions\\
% %Demo\\
% ALPS - startup\\
% Mobile AR\\
% NIST firefighter localization with smart EXIT signs with TI


%Being able to locate people and things inside a building accurately and instantly in a cost-effective  manner  will  revolutionize  the  way  we  interact  with  our  indoor  surroundings  and open up new application domains ranging from location-based authentication, drone navigation indoors to future intefaces using mixed reality. 

%These applications range from indoor navigation, search and rescue, location-based authentication, smart buildings to future interfaces using mixed reality. However, indoor localization at scale is not yet a reality due to two reasons. First, the technical barriers in terms of what sensors are available on commodity devices, and second, indoor localization faces the  typical CPS challenges in scaling up from labs to real-world. 
%Indoor localization at scale is not yet a reality due to technical barriers in  terms  of  what  sensors  are  available  on  commodity  devices  today,  and  gaps  in  scaling these systems from labs to realistic building environments.  
%
%I believe that to enable indoor localization to become a reality, we should embrace the emerging ranging and localization technologies, and solve the problems that these systems face which scaling up. My work takes a principled approach to solving the problems faced by emerging ranging and localization technologies while scaling up, with regard to sensor placement, mapping and noisy sensormeasurements.  More generally, I apply tools from estimation theory and statistical signalprocessing to real-world embedded sensing systems.

%\section{Research in other areas of CPS}
%In addition to localization, I have worked on problems in the area of time-synchronization, energy monitoring and communication in context of CPS. 




\newpage
\small
\bibliographystyle{abbrv}

\bibliography{references}  % sigproc.bib is the name of the Bibliography in this case

\end{document}
